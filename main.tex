\documentclass[supercite,gbt15cite,notofont]{HustGraduPaper}

\title{面向松散RISC-V多核的片外访存架构设计与实现}
\author{刘一鸣}
\school{计算机科学与技术}
\classnum{ZY1801}
\stunum{U201814489}
\instructor{邵志远}
\date{2022年6月1日}

\usepackage{algorithm}
\usepackage{enumitem}
\usepackage{amsmath}
\usepackage{subcaption}
\usepackage{svg}
\usepackage{graphicx}
\usepackage{xurl}

\newcommand{\csvgfig}[3]{
  \begin{figure}[htb]
    \center
    \includesvg[width=#3\columnwidth]{images/#1.svg}
    \caption{#2}
    \label{fig:#1}
  \end{figure}
}

\graphicspath{ {./images/} }

\newcommand{\cpngfig}[3]{
  \begin{figure}[htb]
    \center
    \includegraphics[width=#3\columnwidth]{#1}
    \caption{#2}
    \label{fig:#1}
  \end{figure}
}

\begin{document}

\maketitle

\statement

\clearpage

\pagenumbering{Roman}

\begin{cnabstract}{第五代精简指令集架构;多核架构;存储一致性;访存架构;智能网卡;现场可编程逻辑门阵列}

智能网卡是计算机网络与体系结构领域近年来出现的新兴概念。根据用户实际的项目需要,智能网卡允许用户
通过可编程的方式将网络计算工作卸载到智能网卡上。在这其中,基于片上系统的智能网卡是发展的一大重要
方向。基于片上系统的智能网卡集成了网络处理器,由网络处理器通过访存操作获取网络数据通路上的报文,
并进行处理。因此,访存架构在设计中扮演了重要的角色。但是传统的计算机访存架构所采用的存储一致性
模型无法有效地利用网络数据包彼此之间相互独立的数据特点,所以在实现中带来了额外的性能开销。

本课题向智能网卡片上系统的设计引入了基于松散存储一致性模型的访存架构设计。通过松散存储一致性模型
较弱的一致性约束,降低多个网络处理器之间因存储一致性约束而收到的性能限制。基于第五代精简指令集架构,
以及多个开源项目,本课题将实现基于松散多核访存架构的智能网卡片上系统搭建,并对所搭建的片上系统进行
仿真测试与基于现场可编程逻辑门阵列的上板验证。

本课题实现了内存架构中的缓存系统搭建,并部分完成了智能网卡片上系统的搭建工作。通过软件仿真与上板
验证,本课题验证了实现的正确性。测试所得到的数据指出,本课题所实现的内存架构有效地提升处理器核在
访存架构中的性能表现。尽管如此,上板测试显示目前的片上系统实现在性能上仍然存在不足,尚且有着较大
的优化空间。

\end{cnabstract}

\begin{enabstract}{RISC-V, Many Cores, Memory Consistency, Memory System, SmartNIC, FPGA}

SmartNIC is an emerging concept and trend of research in computer network and
architecture fields. It allows user to practically offload network computing
tasks through a programmable approach. Among the existing work,
System-on-Chip-based (SoC-based) SmartNIC plays an important role. Network
Processors (NP) are integrated into SoC-based SmartNIC, access and handle
on-path network packets by memory accessing operations. turning memory system
to be a key component of SoC-based SmartNIC designs. However, the memory
consistency model traditional computer architectures cannot make efficient use
of the data trait of being independent between two different network packets,
thus brings unnecessary performance cost.

In this paper, we introduce a memory system using Relaxed Memory Order (RMO)
model to SmartNIC SoC design. By implementing the RMO memory consistency model,
the architecture aims to leveraging the cost of keeping strict memory
consistency in traditional memory systems. Based on RISC-V Instruction Set
Architecture (ISA) and many open source projects, this paper builds a SoC-based
SmartNIC using RMO model. Simulator-based and FPGA-based tests are also
performed for verification purposes.

This paper demonstrates the building of cache system, and partly the SoC-based
SmartNIC. By both simulator- and FPGA-based tests, we give a proper verification
of our implementation. The cache system given in this paper significantly
improves the performance of processors in the memory system. However, a
FPGA-based test has shown a poor-functioning in current SoC design, indicating
a larger space for improvement.

\end{enabstract}

\tableofcontents[level=2]

\clearpage

\pagenumbering{arabic}

% This chapter is NOT FINISHED yet, please remeber to finish it later!

\section{绪论}

本章将对论文选题中所提到的相关概念进行介绍,并分析其当今的研究现状,以及学术界与工业界在相关领域
所面临的问题与挑战。本章同时会对本课题的主要研究内容及其意义进行阐述。

\subsection{课题背景}

\subsubsection{RISC-V开放指令集}
\label{section:background_riscv}

RISC-V开放指令集\cite{waterman2016design, waterman2014risc, waterman2015risc}又称
第五代精简指令集,是近年来兴起的全新的、简单且开放自由的指令集架构。其于2010年在美国加州大学伯克利
分校(University of California, Berkeley,UC Berkeley)的Krste Asanovic教授、Andrew
Waterman教授和Yunsup Lee教授等人所开发,并得到了David Patterson教授的大力支持。不同于其
前辈如x86和ARM等商业指令集,标榜开放的RISC-V指令集可以被任何的学术机构或是商业组织自由使用。
得益于这样的开放特性,RISC-V指令集一经问世就受到了来自学术界和工业界的欢迎。目前,RISC-V指令集
已经得到了诸如Intel、华为等大型科技公司的支持\cite{riscvmember}。

\subsubsection{多核处理器架构}
\label{section:background_manycore}

多核处理器架构这一概念是相对于单核处理器架构而言的。相较于单核处理器架构仅在架构中集成单个处理器核
的设计而言,多核处理器架构中会包含不止一个处理器核。随着制造工艺的发展,复杂芯片设计面临着越来越多
的物理因素制约,想要在单个处理器上取得更好的性能表现正变得越来越困难。同时,单个处理器核也不能
很好地提升那些符合线程级并行(Thread-Level Parallelism,TLP)的应用的性能。

而多核处理器架构能够通过多个处理器核,引入新的并行度,进而提升架构整体的性能。多个处理器核的设计
也天然地适合于提升线程级并行应用的性能。因此,多核架构的设计成为了计算机体系结构领域近些年来的
热门研究方向。

\subsubsection{存储一致性模型}
\label{section:background_consistent_model}

但多核处理器架构的设计同样也引入了新的设计上的复杂性。这其中最重要的一点就是多个处理器核之间的
存储一致性问题。为了解决这一问题,学术界引入了多种存储一致性模型。存储一致性模型定义了多核系统
访问存储器时所需要遵守的约束。就约束的强弱而言,存储一致性模型能够分为顺序一致性模型(Sequential
Consistency,SC)、完全存储定序模型(Total Store Order,TSO)和松散一致性模型(Relaxed
Memory Order,RMO)\cite{sorin2011primer}。通常来说,存储一致性模型所施加的约束越强,则
越容易保证访存操作的正确性,但性能也相应的容易受到限制;反之,所施加的限制越弱,则越有利于提升
性能,但访存操作的正确性也就越难保证。

在现代的处理器架构设计中,为了同时兼顾性能与访存操作的正确性,处理器架构设计者们通常会选择一个
较为折中的存储一致性模型来为处理器架构提供存储一致性约束。如x86架构的处理器就使用了完全存储定序
模型,而RISC-V架构的设计者则使用了偏向于松散一致性模型的“RISC-V弱内存一致性模型(RISC-V Weak
Memory Order,RVWMO)”\cite{waterman2016design,waterman2014risc,waterman2015risc}。

\subsubsection{智能网卡}
\label{section:background_smartnic}

网络接口控制器(Network Interface Controller,NIC)是一种用于连接到以太网网络的PCIe卡,也被称为网卡。
传统的NIC实现了主机到以太网网络的连接。而在传统NIC的基础上,智能网卡(SmartNIC)不仅提供了
连接,其同时还实现了传统架构下CPU必须执行的网络流量处理功能。基于这样的特性,现有的工作给出了
关于智能网卡的定义:能够卸载CPU通用处理任务的网卡\cite{pcmag_smartnic,maxiaoxiao2022survey}。

部分工作已经指出,通过CPU访问内存进行数据移动的开销在很多应用中占到了极大的比例\cite{maxiaoxiao2022survey}。
同时,随着后摩尔时代的到来,CPU的计算频率已经日渐落后于快速发展的网络速度。这都使得单纯依靠CPU
的网络计算所得到的性能表现越发地不尽如人意。因此,能够满足高速网络处理需求智能网卡受到了来自学术界
与工业界的密切关注。

依照智能网卡的实现方式,智能网卡通常可以分为三大类\cite{bhalgat2021smartnic}。

\begin{enumerate}
  \item 基于ASIC的智能网卡。这类网卡基于ASIC设计,通常具有优秀的性能表现,且成本相对低廉。但其
        开发成本高,可编程性也仅局限于预定义的固定功能;
  \item 基于FPGA的智能网卡。这类网卡同样拥有良好的性能表现,但通常价格昂贵。基于FPGA的设计带来
        更加丰富的可编程性,但FPGA开发仍然存在着一定的难度;
  \item 基于SoC的智能网卡。这类网卡可以使用网络处理器(Network Processor,NP)来提供可编程性,
        因此具有三类网卡中最好的可编程性与灵活性,但网络处理器的性能则难以与基于ASIC或是FPGA的
        设计相比拟。
\end{enumerate}

\subsection{国内外研究现状}

\subsubsection{RISC-V处理器}

如其名字,RISC-V指令集是一种基于精简指令集(Reduced Instruction Set Computer,RISC)的
指令集架构。这使得RISC-V指令集比起x86这样的复杂指令集(Complex Instruction Set Computer,
CISC)而言在设计上更为精简整齐。同时,RISC-V指令集充分吸收了指令集设计的历史经验,作为一款年轻
的指令集也可以摆脱如x86或ARM中的许多历史包袱。

RISC-V指令集采用模块化设计,除了最为基本的整数指令集外,RISC-V指令集根据不同目的被划分为多个
拓展指令集,使得厂商在设计与生产时可以根据需要自由组合。目前常被使用的拓展指令集如
\autoref{tbl:riscv_extension}所示。

\begin{generaltab}{常用RISC-V拓展指令集}{tbl:riscv_extension}
  \begin{tabular}{cc}
    \toprule
    指令集简称 & 指令集表述 \\
    \midrule
    I & 整数指令集 \\
    M & 整数乘除法拓展 \\
    A & 原子指令拓展 \\
    F & 单精度浮点型运算拓展 \\
    D & 双精度浮点型运算拓展 \\
    C & 压缩指令拓展 \\
    Zicsr & 控制状态寄存器拓展 \\
    Zifence & FENCE指令拓展 \\
    V & 向量指令拓展 \\
    H & Hypervisor拓展 \\
    \bottomrule
  \end{tabular}
\end{generaltab}

RISC-V指令集灵活的模块化设计使得其可以根据需要被应用到各种场景。从简单的工控MCU,到个人桌面PC,
甚至是大型计算机。

而RISC-V指令集的开放性也使其得到了学术界与工业界的支持。而其中也不乏影响力深远的项目。例如加州
大学伯克利分校的RocketChip处理器\cite{asanovic2016rocket}。RocketChip处理器是RISC-V开放
指令集下的经典实现,其提供了RV64GC指令集的良好支持,并使用加州大学伯克利分校所开发的敏捷开发语言
Chisel所实现。目前已经有多个开源项目应用了RocketChip处理器作为研究对象\cite{rigge2018designing,
wang2021instruction,wei2020evaluation}。而除了RocketChip外,加州大学伯克利分校还推出了
针对高性能处理器的BOOM(Berkeley Out-of-Order Machine)处理器\cite{zhao2020sonicboom}。
BOOM处理器是面向高性能领域的乱序三发射处理器,同样支持RV64GC指令集。在CoreMark测试中,BOOM
能够达到RocketChip将近两倍的性能评分\cite{zhao2020sonicboom}。

而除了加州大学伯克利分校开发的两款处理器外,面向不同的应用常见与开发目的,海外的其他高校与企业也
纷纷推出了自己的RISC-V处理器核设计。如lowRISC推出的面向低功耗场景的ibex处理器\cite{lowrisc_ibex}、
OpenHWGroup推出的兼顾能耗和性能的CVA6处理器\cite{zaruba2019cost},以及针对面积优化的
picorv32处理器\cite{picorv32}。

而对于国内的RISC-V指令集生态俄而言,目前比较有影响力的项目则是有中科院计算所牵头设计的“香山”
高性能处理器\cite{xiangshan_riscv}。“香山”高性能处理器采用了超标量处理器设计,且同样提供了
RV64GC指令集的支持。目前,“香山”处理器已经通过了一轮流片迭代,并能够成功运行较为Linux操作系统。
作为国内为数不多经过流片验证的开源超标量处理器设计,“香山”处理器在国内有着显著的技术地位,是国内
开源芯片生态中的重要一环。

\subsubsection{多核处理器架构}

对于RISC-V开放指令集而言,多核处理器架构同样是生态中非常重要的一环。作为最早实现了RISC-V开放
指令集的处理器设计,RocketChip在问世之初就提供了多核架构的设计,并提出了Tilelink总线协议来
支持RISC-V弱内存一致性模型\cite{asanovic2016rocket}。而除了加州大学伯克利分校的工作外,如
lowRISC和苏黎世联邦理工大学也发布如Ariane、PULP这样的工业级低功耗多核SoC设计\cite{balkind2019openpiton+,
rossi2017energy}。同时,国内的企业与高校对于多核架构的设计也有着相当的研究\cite{manycore2013fudan,
gs464e2015recent},一些工作还深入地研究了多核处理器架构对于在当今时代对于软件设计者影响
\cite{manycoretime2016recent}。

\subsubsection{智能网卡}
\label{section:present_smartnic}

在智能网卡出现之前,传统网卡一直是计算机网络领域关注的焦点之一。在长期的发展中涌现出了不少颇具影响力
的企业。如TP-Link就是以提供网络设备而闻名的企业,其提供了多种PCIe网卡以应对不同的应用场景需求
\cite{tplink_website}。而国内的网络设备生产厂商则是以华为为代表\cite{huawei_website}。

但受限于商业公司的不开放性,商业公司为了保持其竞争力大多不会将所研发的网卡设计开源。因此长期以来
传统网卡领域一直缺少具有影响力的开源网卡实现。这一现象直到Corundum\cite{forencich2020corundum}
的出现才有所改观。Corundum是加州大学圣地亚哥分校所研发的基于FPGA的网卡平台,并在测试中最高能达到
100Gbps的优异性能。

而对于智能网卡,在\autoref{section:background_smartnic}中提到,智能网卡按实现方式可以分为三类。其中,
基于FPGA的智能网卡实现方案在学术界主要以微软研究院为代表\cite{firestone2018azure,fowers2015scalable,
caulfield2016cloud}。微软在Catapult项目的基础上,将FPGA、CPU与数据中心网络深度融合,在SmartNIC
领域作出了大量的工作。而部分老牌公司如Intel、Mellanox、Xilinx等也陆续推出了许多基于FPGA的智能
网卡产品\cite{mellanox2020whitepaper,intel2020pac,intel2019d5005,xilinx2020alveou25,
xilinx2020x2}。

基于SoC的智能网卡设计则是近年来受到业界广泛认可的另一种设计模式。SoC中集成的处理器核为智能网卡
带来了极为灵活的可编程性。结合现有的通用处理器(General Processor,GP),Mellanox推出了基于
BlueField IPU系列可编程智能网卡,其上集成有ARM处理器阵列,可以运行如Ubuntu、CentOS等主流Linux
发行版\cite{mellanox2020bluefield}。除此之外,业内的其他厂商也已经推出过基于通用处理器/网络
处理器(Network Processor,NP)的智能网卡产品\cite{broadcom2020stingray,annapurna2020announce}。
除了工业界的产品以外,学术界也在基于SoC的智能网卡设计领域作出了不少的工作\cite{di2020pspin}。

目前基于ASIC的智能网卡产品则相对较少。ASIC芯片更多地则是以网络控制器的角色出现在智能网卡中,用于
满足传统的网络协议处理需求,并提供一定的卸载CPU处理能力和可编程性\cite{maxiaoxiao2022survey}。

\subsubsection{面临的问题与挑战}
\label{section:problems_and_challenges}

\autoref{section:background_manycore}与\autoref{section:background_consistent_model}
指出基于多处理器核的处理器架构设计能够通过增加处理器核的数量来增加架构中的并行程度,从而提升整体
的性能。但是,在常见的应用场景(如桌面PC、服务器等)下,为了保证访存操作的的正确性,架构设计者
在设计时通常会遵循特定的存储一致性模型。存储一致性模型的存在尽管保证了访存操作的正确性,但其同时
也在一定程度上限制了处理器架构的性能提升。

但在智能网卡的应用场景中,这样的约束是可以变得宽松的。在网络计算的应用场景下,因为数据包之间存在
着较强的独立性,因此负责对数据包进行处理器的处理器核在访问数据时的访存操作之间也具有较强的独立性。
这样的独立性使得智能网卡在设计所搭载的多核处理器架构时可以采用较为松散的存储一致性模型。对于基于
SoC的智能网卡设计而言,这样的应用特性显然带来了潜在的性能提升空间。

尽管如此,采用松散的存储一致性模型也会带来额外的设计上的复杂。为了避免因为宽松的存储一致性模型而
导致与访存操作有关的错误,必须尤为慎重地设计多核处理器架构的数据通路和访存模块,以避免潜在的访存
冲突。尽管目前已经存在一定数量的关于松散内存一致性模型的研究,但针对智能网卡这一应用场景的研究
尚且数量较少。

\subsection{课题研究的意义、内容和目标}

\subsubsection{课题研究的意义}
\label{section:paper_making_sense}

如\autoref{section:background_smartnic}所提到的,目前智能网卡主要分为三个发展方向:基于FPGA的
智能网卡设计、基于SoC的智能网卡设计以及基于ASIC的智能网卡设计。其中,基于SoC的智能网卡往往使用
大量的嵌入式CPU处理器核来对数据包进行处理。这样的做法带来了编程性上的优势,但相对的其性能通常低于
基于FPGA和基于ASIC的智能网卡设计。目前,工业界所设计的基于SoC的智能网卡往往选择使用基于ARM指令集
的嵌入式处理器核作为智能网卡的处理器核。考虑到ARM指令集在嵌入式领域的地位,工业界的选择显然是可以
理解的。

然而,RISC-V开放指令集的兴起为智能网卡的设计者们提供了新的选择。尽管作为新兴的指令集架构,但当下
也已经涌现出了一大批优秀的RISC-V处理器核设计,其中更是不乏面向嵌入式领域的优秀设计。在部分研究
中,基于RISC-V指令集的设计甚至能取得优于现有ARM处理器核的性能、能耗与面积(Power, Performance
and Area,PPA)\cite{asanovic2014instruction}。同时,作为开放的指令集架构,RISC-V指令集
在使用时更加自由,不需要如ARM指令集一样受到诸多商业上的限制。尽管如此,目前学术界与工业界将RISC-V
指令集应用到智能网卡设计的案例还停留在实验阶段。但考虑到RISC-V指令集的各种优势,本课题将尝试使用
RISC-V处理器核作为智能网卡的处理器核,借此探究RISC-V指令集在智能网卡设计上的应用。

同时,现有的基于SoC的智能网卡设计尽管大量使用了多核架构设计,但对于松散存储一致性模型的研究却相对
较少。一些开源的智能网卡实现往往会要求一定的存储一致性约束,并通过频繁的数据移动(通常是通过直接
内存访问(Direct Memory Access,DMA)操作来实现)来保证处理器所访问的存储器数据的正确性\cite{di2020pspin}。
频繁的数据移动尽管保证了处理器核访存操作的正确性,但也带来了额外的数据移动开销,影响了智能网卡性能的
进一步提升。

而本课题则认为,在智能网卡的包处理应用场景下,各个处理器核之间的访存操作相对独立,依赖性不强,适合
使用松散存储一致性模型。松散存储一致性模型将会避免因频繁的数据移动而带来的性能影响,从而进一步地
提升智能网卡的性能。本课题将会探究松散存储一致性模型在智能网卡多核架构设计下的可行性,并借由
本课题中的实现,探究松散存储一致性模型对于智能网卡的性能影响。

\subsubsection{课题研究的内容}

如本课题标题所示,“面向松散RISC-V多核的片外访存架构设计与实现”。本课题的内容专注于多核架构的设计
与实现。本课题将探究松散RISC-V多核架构下的片外访存架构的设计与实现,通过多核架构的设计,实现基于
松散存储一致性模型的访存架构。多核架构中所集成的RISC-V处理器核将能够通过访存架构正确地访问片外的
存储器资源,如动态随机访问内存(Dynamic Random Access Memory,DRAM)与高带宽内存(High-
Bandwidth Memory,HBM)。

本课题同时还将探究所设计并实现的多核架构在智能网卡应用场景下的应用。通过将所设计和实现的多核架构
集成进已有的开源网卡实现中,为网卡提供可编程性,使其符合智能网卡的定义。本课题将探究这样的集成改造
验证松散多核架构在智能网卡包处理应用场景下的可行性,并对比研究松散多核架构对于智能网卡性能的影响
情况。

\subsubsection{课题研究的目标}

本课题的研究内容主要围绕着松散RISC-V多核访存架构展开。以开源的RISC-V处理器核为基础,本课题将
搭建支持松散存储一致性模型的多核架构,并对架构实现进行验证与性能测试。同时,本课题将基于现有的开源
网络接口控制器设计,对多核架构进行集成,使网络接口控制器获得可编程性。在此基础上,对设计进行验证
与性能测试。探究松散多核架构对于网络接口控制器的性能影响。

\subsection{论文结构}

本论文将分为如下若干部分分别展开。在第二章中,将对本课题的需求进行分析,并结合前期调研内容,论证
本课题在工程实践上的可行性。在第三章中,将对本课题项目的整体设计进行阐述,并明确架构中部分关键部件
的设计思路。第四章则会重点描述本课题在工程上对于架构的实现方案,并对各类技术细节进行说明。第五章
将列举本课题目前所做的测试工作,并对各类测试结果进行分析。最后,第六章将会对全文进行总结,并介绍
下一步的工作方向。

\section{方案论证}

本章首先对本课题背后的系统需求进行描述,说明本课题的现实意义。随后,本章将对系统需求进行分析,分析
满足系统需求地可行性。紧接着,本章将描述本课题所用到的各种技术开发工具,并解释这些开发工具在开发
过程中所扮演的角色。最后,本章同时会分析给出基本的技术方案,并罗列其中所设计的关键技术。

\subsection{系统需求分析}

随着现代计算机网络技术的不断发展,计算机网络带宽正不断提升。而随着这样的提升,对计算机网络接口
控制器的要求也正在不断提升。而对于网络接口控制器的一种改进是将原本由主机完成的部分计算工作卸载到
网络接口控制器上,从而减少主机与网络接口控制器之间的数据移动,从而实现网络处理性能的提升。

这样的优化思路促成了智能网卡这一概念的诞生。智能网卡允许用户通过可编程的方式,有目的性地将部分计算
工作卸载到智能网卡上。这样的可编程性使得智能网卡能够适应新的协议与政策的发展进化,并积极对接
快速更新的上层软件。

基于SoC的智能网卡是目前智能网卡发展的一大重要方向。通过智能网卡上所集成的嵌入式处理器核,智能网卡
能够对经过网卡的网络数据包进行处理。同时,接近于通用处理器的嵌入式处理器核则能提供最为接近软件环境
的编程模型。这使得基于SoC的智能网卡通常拥有良好的可编程性。

在基于SoC的智能网卡设计上,访存架构扮演着重要的角色。好的访存架构能够减少智能网卡上的处理器核执行
访存操作所消耗的时间,从而加快数据处理的速度。\ref{section:paper_making_sense}节中讨论了当前
网络数据包处理的应用场景下的相关特点,数据包之间的独立性使得核与核之间的访存操作不再具有严格的
依赖关系。这样的应用场景使得松散多核访存架构成为了提升SoC访存操作性能的潜在方向。

\subsection{系统可行性分析}

基于SoC的智能网卡设计是已经经过了工业界验证的智能网卡解决方案。在\ref{section:present_smartnic}节
中提到了多款已有的基于SoC的智能网卡实现。来自老牌工业厂商的产品表明了工业界对于这一方向的兴趣,
一定程度上也表明了这一方向的可行性。同时,如今也存在不少来自学术界的开源实现可供参考,如PsPIN\cite{di2020pspin}等。

学术界对于松散多核访存架构也已经有着相当的研究,如松散存储一致性模型这一概念也是学术界有所涉足的
研究领域\cite{adve1996shared}。现有的RISC-V指令集架构,采用的就是被称为“RISC-V弱内存一致性
模型”的松散存储一致性模型\cite{waterman2014risc}。因此,RISC-V指令集本身对于松散多核访存架构
就有着良好的适应性,适合于搭建本课题所要求的松散多核访存架构。

\ref{section:principle_tools}节将会详细介绍用于开发与测试的各种软硬件工具。Verilog等硬件
描述语言为项目提供了成熟的设计开发方案。而现有的如Verilator等开源的仿真测试工具则可被用于本课题
的仿真测试。同时,Verilator为本课题提供了基于C/C++的高层次建模方案,使得可以在进行实际的硬件
开发之前先通过高层次行为仿真的方式对设计进行验证。

\subsection{开发工具分析及选择}
\label{section:principle_tools}

\subsubsection{Verilog硬件描述语言}

% The Verilog language is a hardware description language that provides a means of
% specifying a digital system at a wide range of levels of abstraction. The language
% supports the early conceptual stages of design with its behavioral level of
% abstraction, and the later implementation stages with its structural abstractions.
% The language includes hierarchical constructures, allowing the designer to
% control a description's complexity.

Verilog语言是一种硬件描述语言(Hardware Description Language,HDL),用于在不同的抽象层面
上描述数字电路系统。Verilog语言提供了行为层面的抽象能力用于应对集成电路设计早期的概念层面的设计,
同时还提供了结构层面的抽象用于之后的具体实现阶段\cite{thomas2008verilog}。Verilog语言所提供
的多种抽象能力使得设计者能够使用不同的复杂度对硬件设计进行描述。

% Verilog was originally designed in the winter of 1983/84 as a proprietary
% verification/simulation product. Later, several other proprietary analysis tools
% were developed around the language, including a fault simulation and a timing
% analyzer. More recently, Verilog has also provided the input specification for
% logic and behavioral synthesis tools. The Verilog language has been instrumental
% in providing consistency across these tools. The language was originally
% standardized as IEEE standard #1364-1995. It has recently been revised and
% standardized as IEEE standard #1364-2001.

Verilog语言最早于1983年到1984年被创造,最早在商业工具中被作为验证与仿真工具使用。该语言最初只作为
仿真验证工具使用,而不是一门用于设计硬件的语言。但随着Verilog语言的推广与大规模应用,围绕Verilog
语言的可综合语法和综合器也陆续被开发出来。这使得Verilog语言真正地拥有了能够在各个抽象层面描述硬件
设计的能力。

Verilog语言在1995年被IEEE标准化,编号为IEEE 1364-1995。这一版本的Verilog语言也通常被称为
Verilog-95。之后,在2001年,IEEE再次修订了此前的Verilog语言标准,并推出了修订后的标准
IEEE 1364-2001,也被称为Verilog-2001。Verilog-2001是目前工业界使用的最为广泛的标准,各大
主流厂商的工具均提供了对这一标准的良好支持。Verilog语言标准的最新版本为2005年通过的IEEE 1364-2005,
也被称为Verilog-2005。这一版的标准对Verilog-2001做了些许修正,并增加了部分新的语言特性。
SystemVerilog语言是建立在Verilog-2005标准之上的Verilog语言的超集。其在Verilog语言的基础上,
增加了大量有助于仿真验证的语法特性。丰富的仿真验证语法使得SystemVerilog语言很快就受到了各大
主流电子设计自动化(Electronic Design Automation,EDA)工具厂商的青睐,并很快为SystemVerilog
语言提供了良好的工具支持。到了2009年,SystemVerilog语言与Verilog语言正式合并成为了SystemVerilog-2009(IEEE 1800-2009)。
SystemVerilog语言成为了事实上Verilog语言的继任者。目前最新的SystemVerilog标准为IEEE 1800-2017。

尽管SystemVerilog-2017是目前最新的Verilog语言标准,但目前工业界在进行集成电路设计时仍然广泛
地使用Verilog-2001作为设计语言标准。一方面Verilog-2001能够得到大部分开发工具的良好支持,另一
方面现有的大量开源项目也普遍将Verilog-2001作为其代码标准。鉴于此,本课题选择使用Verilog-2001
作为最基本的开发语言。

\subsubsection{Verilator高性能仿真工具}

% Verilator is invoked with parameters similar to GCC or Synopsys's VCS. It
% "Verilates" the specified Verilog or SystemVerilog code by reading it,
% performing lint checks, and optionally inserting assertion checks and coverage-
% analysis points. It outputs single- or multi-threaded .cpp and .h files, the
% "Verilated" code.

% The user writes a little C++/SystemC wrapper file, which instantiates the
% "Verilated" model of the user's top level module. These C++/SystemC files are
% then compiled by a C++ compiler (gcc/clang/MSVC++). The resulting executable
% performs the design simulation. Verilator also supports linking its generated
% libraries, optionally encrypted, into other simulators.

Verilator高性能仿真工具是一种类似于GCC或是新思科技的VCS的仿真工具。它能够对所提供的Verilog
或是SystemVerilog代码进行语法检查和仿真测试,并且能够根据选择插入断言和覆盖率分析语法。
Verilator高性能仿真工具能够将Verilog/SystemVerilog代码转换为单线程或是多线程的C++代码,并
通过C++编译器(如GCC、CLANG或是MSVC++)将其编译为可执行程序进行仿真\cite{verilator_website}。

Verilator并不会直接将Verilog代码翻译成C++或是SystemC代码。相反,其会将代码编译为一种经过优化
且拥有更好性能的多线程模型。使用Verilator对Verilog代码进行编译得到的代码在性能表现上能够接近
手工编写的仿真代码\cite{snyder2017verilator},而使用Verilog显然也更容易上手开发。对比已有
的仿真工具而言,Verilator也能取得更好的仿真速度\cite{snyder2013verilator}。

更重要的是,Verilator是现有的少数开源仿真工具。目前在工业界所使用的大多数仿真测试工具都需要支付
昂贵的使用费用,并且通常需要在服务器上部署,使用起来较为复杂。而作为开源软件,Verilator不需要
商业证书,同时也可以直接在笔记本等一类个人电脑上运行,使用简单快捷。

\subsubsection{Xilinx Vivado设计工具组}

Vivado设计工具组是Xilinx公司提供的用于对HDL设计进行综合与分析的软件工具组。Vivado最早于2012
年推出,并作为系统到集成电路级别的工具所使用。Vivado设计工具组主要包含如下四个部分

\begin{enumerate}
  \item Vivado高层次综合(High-Level Synthesis,HLS)工具。这类工具允许使用C/C++以及
        SystemC进行开发,将软件代码直接映射到Xilinx的设备上,而不需要手工地进行寄存器传输级(
        Register-Transfer Level,RTL)代码开发。Vivado高层次综合工具经过了大量验证,且
        能够良好地支持C++中的类、模板类、函数和运算符重载等语法。使用高层次综合工具进行代码开发
        被认为能够大幅度降低硬件开发难度,降低设计人员的开发成本。
  \item Vivado仿真器是Vivado设计工具组中的另一重要组成部分。作为编译型仿真软件,Vivado仿真器
        能够提供对于多种硬件描述语言的良好支持,如Verilog、SystemVerilog以及VHDL。此外,
        Vivado仿真器还支持如TCL脚本、加密IP集成以及增强型验证等功能。
  \item Vivado IP集成工具允许工程师能够快速地集成与配置来自于Xilinx IP库中的IP。通过这一工具,
        负责集成设计的工程师能够更好地对大型片上系统中的IP核进行管理,从而加快整个系统的集成速度。
  \item Vivado TCL Store是Vivado设计工具组中用于开发附加组件的脚本系统。这一系统使用TCL语言
        作为脚本语言,且允许用户通过TCL脚本控制大部分的Vivado底层功能。借由TCL脚本,用户可以
        更好地对设计开发流程进行脚本化与流程化,从而减少开发过程中的流程错误,加快项目整体的开发
        流程。
\end{enumerate}

Vivado设计工具组提供了Xilinx 7系列及此后新产品(如UltraScale与UltraScale+系列)的工具支持。
因此,对于本课题的FPGA验证部分将依赖于Vivado设计工具组展开。

\subsubsection{Xilinx Alveo U280数据中心加速卡}

Xilinx Alveo U280数据中心加速卡是Xilinx公司所推出的一款数据中心数据加速卡。其支持PCIe接口,
且配备有8GB内存大小的高带宽内存(High-Bandwidth Memory,HBM),以及两块16GB内存的DDR4 RDIMM
存储器。在网络接口方面,Xilinx Alveo U280提供了两个QSFP28以太网接口。Xilinx Alveo U280
加速卡被设计用于加速内存密集型、计算密集型应用,如数据库分析以及机器学习应用等\cite{alveo2021datasheet}。
该计算加速卡部分参数如\autoref{tbl:alveo_datasheet}所示。

\begin{generaltab}{Xilinx Alveo U280加速卡部分参数\cite{alveo2021datasheet}}{tbl:alveo_datasheet}
  \begin{tabular}{cc}
    \toprule
    参数名 & 具体配置 \\
    \midrule
    网络接口 & 2x QSFP28 \\
    PCIe接口 & Gen3 x16, Gen4 x8, CCIX \\
    HBM2总容量大小 & 8 GB \\
    HBM2总带宽 & 460 GB/s \\
    LUT数量 & 1,304K \\
    寄存器数量 & 2,607K \\
    Block RAM数量 & 2,016 \\
    UltraRAM数量 & 960 \\
    DDR总容量 & 32 GB \\
    DDR最大数据速率 & 2400 MT/s \\
    DDR总带宽 & 38 GB/s \\
    \bottomrule
  \end{tabular}
\end{generaltab}

可以看出,Xilinx Alveo U280加速卡提供了非常丰富的网络资源可供使用。同时,其上所搭载的高带宽
存储器(HBM、DDR)也提供了充足的内存空间用于大型应用使用。HBM2所提供的高带宽则充分地保障了其上
所搭载应用的访存性能,并提供了足够的优化空间。

\subsubsection{Python脚本语言}

Python是一种简单易学但功能强大的编程语言。其提供了高效的高层次数据结构以及简单明了的面向对象编程
(Object-Oriented Programming,OOP)能力\cite{python_tutorial}。Python的语法本省贴近
于自然语言表述,这使得Python语言很便于学习与编写。同时,Python简单易用的特点也适合被用于编写
各类脚本以及在各个领域进行快速的应用开发工作。

Python语言属于解释执行语言。借助于不同平台下的Python语言解释器,所编写的Python代码可以很无需
太多修改就在不同的平台上运行。同时,出于性能上的优化考虑,Python解释器也可以很方便得通过C/C++
(或是其他支持C ABI的语言)进行性能优化。这大大加速了Python在某些复杂应用下的性能表现。在拓展
性上,Python也可以很轻松地被其他语言所拓展,或是作为某种特定应用场景下的拓展语言使用\cite{binkert2011gem5}。

如上述所言,鉴于Python语言的种种优点,Python已经被广泛地应用于各种领域。在HDL开发领域,不少项目
会选择使用Python脚本来完成一些诸如语法格式检查、代码修改等工作。此外,也存在基于Python的HDL测试
框架可供使用\cite{cocotb_doc}。

\subsubsection{Corundum}
\label{section:principle_corundum}

Corundum是一款基于开源的网络接口控制器实现。其被设计作为针对网络接口技术开发的基于FPGA的实验性
平台。Corundum在设计时考虑了如下几个关键功能\cite{forencich2020corundum}。

\begin{enumerate}
  \item 高性能的数据通路;
  \item 10G/25G/100G的以太网媒体接入控制(Media Access Control,MAC)端口;
  \item 第三代PCI Express接口;
  \item 定制的PCIe DMA引擎;
  \item 原生高精度IEEE 1588 PTP时间戳;
  \item 支持分散/聚集(Scatter/Gather)功能的DMA引擎、校验核卸载以及接收流哈希等功能;
  \item 支持消息信号中断(Message Signal Interrupt,MSI)机制;
  \item 支持可配置数量的发送队列()、接收队列、发送完成队列、接收完成队列和事件队列;
  \item 提供了Linux网络协议栈的良好支持。
\end{enumerate}

实验数据指出,Corundum能够达到100Gbps甚至更高的性能表现\cite{forencich2020corundum},
这在现有的开源网卡中是非常难得的。

Corundum的关键架构如\autoref{fig:corundum_datapath}所示。出于绘图的美观起见,图中部件均
使用英语缩写标明,其对应全称解释如下:PCIe(Peripheral Component Interconnect Express,
高速串行计算机拓展总线接口)、TXQ(Transmit Queue,发送队列)、RXQ(Receive Queue,接收队列)、
TXCQ(Transmit Complete Queue,发送完成队列)、RXCQ(Receive Complete Queue,接收完成
队列)、Sched(Scheduler,调度器)、Cpl(完成写入模块)、Engress(数据包输出)、Ingress
(数据包输入)、FIFO(First-In-First-Out,先入先出缓冲队列)、MAC(Media Access Control,
媒体接入控制端口)。

\csvgfig{corundum_datapath}{Corundum网络接口控制器关键框架图}{1.0}

从\autoref{fig:corundum_datapath}可以看出,Corundum的数据通路主要分为发送通路与接收通路
两条路径。这两条路径分别对应网络数据包的发送与接收工作。连接Corundum的主机通过PCIe总线访问发送
通路与接收通路上的数据,从而实现网络交互功能。在Corundum内部,各数据通路上的部件则主要是通过
AMBA总线协议\cite{arm2011axi}进行数据传输。

总得来说,Corundum使用Verilog语言进行开发,其代码本身简单易读;同时其本身的设计较为简单易懂,
数据通路清晰,内部大多使用AXI Lite等常用总线,便于进行改造和连接。因此很适合作为本课题的实现平台。

\subsection{基本方案指定}

% 结合\ref{section:principle_corundum},本课题选择在Corundum的基础上进行开发、修改。这是因为
% Corundum的数据通路较为简单清晰,便于进行修改,增加数据包处理逻辑。在Corundum的基础上,
% 本课题的技术方案将主要分为两个部分:软件仿真与硬件开发。软件仿真部分包括使用C++、Verilog等编程
% 语言,结合Verilator等仿真工具,对项目设计进行高层次的建模与仿真验证。

% 基本的技术修改方向
% 在Corundum基础上进行修改
% 使用Verilator进行仿真验证
% 使用Alveo U280开发板进行测试

TODO

\subsection{关键技术分析}

TODO

\subsection{本章小结}

TODO

\section{松散RISC-V多核访存架构设计}

本章主要对本课题的内容——松散RISC-V多核访存架构——的设计进行介绍。本章首先阐述本设计所面对的功能
需求进行阐述。随后,本章将会描述架构的整体设计,以便能展示架构的大致样貌。最后,本章将会详细解释
架构中部分关键部件的设计。

\subsection{功能需求}
\label{section:design_need}

如\autoref{section:background_smartnic}所言,本课题的主要需求来自于智能网卡上的对于数据包
处理的多核架构需要。已经提到,传统应用场景下所使用的较为严格的存储一致性模型并不能很好地利用智能
网卡应用场景下的数据特点,因此会带来潜在的性能损耗。而使用松散存储一致性模型或许能更好地利用这些
数据特点,实现性能上的提升。

同时,智能网卡这类入网计算的应用场景也需要一套不同于传统应用场景下的访存架构,以便能满足对于高效
网络数据包处理的需求。在这样的访存架构中,负责计算工作的网络处理器(Network Processor,NP)
将会需要一套访存架构来访问数据通路上所传输的网络数据包,并对这些数据包进行处理。

\subsection{系统总体设计}
\label{section:design_overview}

\subsubsection{针对网络数据通路的访存设计}

如\autoref{fig:corundum_datapath}所示,Corundum上的数据在与外部网络进行交互时,会通过端口
上的缓冲队列结构,再离开/进入媒体访问控制器。而为了实现对数据通路上数据包的访问,就需要设计实现
处理器核对数据缓冲队列的访存通路设计。

\csvgfig{design_datapath}{针对网络数据通路的访存设计}{0.8}

\autoref{fig:design_datapath}展示了本课题数据通路设计的基本结构。其中,Picorv32构成的处理器
核将作为网络处理器(Network Processor,NP)使用。同时图中还展示了系统中其他可供网络处理器访问
的存储单元。其中,SRAM模块将为网络处理器提供内存用于堆栈逻辑地址空间的数据存储,并用于存放网络处理器
的部分运行时代码;而CTRL模块则属于网络处理器的控制单元,用于控制网络处理器对数据包的处理流程。
后续的章节中将会对这两个模块进行更为细致的描述。

XBAR是一种常见的片上互联结构(Interconnect)。通常,这样的结构被用于连接片上的各个总线单元,
并对总线上的数据传输进行地址译码与分发。在本课题中,使用XBAR结构来连接各个网络处理器、Corudnum
缓存队列以及片外的DDR等存储器件。使用XBAR结构将可以对多个网络处理器所发起的总线传输操作进行协调
与调度,以保证每个网络处理器都能够实现对存储器件的访问。

图中的FIFO即代表了Corundum上的数据包缓冲队列结构。在具备常规缓冲队列所具备的先进先出功能的同时,
缓冲队列还提供了类似于内存的访问方式。缓冲队列所提供的一整套接口允许外部的网络处理器通过访存指令
对其进行访问,以便对其上的数据进行操作。

DDR代表了外部的双倍数据率同步动态随机存取存储器(Double Data Rate Synchronous Dynamic
Random-Access Memory,DDR SDRAM)。这是一种拥有较高速率的片外存储设备,且往往具有较大的数据
容量。在设计中DDR将用于存储供网络处理器运行的嵌入式软件程序,并同时保存一些占用空间较大的数据结构。
DDR同样通过XBAR互联结构与访存架构中的其他部件相连,以确保每一个网络处理器都能顺利地访问存储在
DDR上的相关数据。

\subsubsection{系统控制通路设计}

控制通路主要用于控制访存架构中的网络处理器的行为。在设计上会使用调度器对所有的网络处理器进行控制,
同时调度器也会负责将缓存队列结构上的数据包分发给各个处理器核用于处理。

\csvgfig{design_controlpath}{系统控制通路设计}{0.9}

一方面,调度器与Corundum的缓冲队列相连。当缓冲队列收到新的数据包时,其就会想调度器发送包含数据包
相关信息的请求(Request)。而调度器收到请求后将会该请求传递给分发器(Dispatcher),再由分发器
将数据包发送给对应的网络处理器进行处理。

另一方面,调度器通过与网络处理器的连接来实现对网络处理器的控制。调度器通过睡眠与唤醒机制,来实现
所需要的控制行为。如果当前不存在数据包共网络处理器处理,那么调度器就会使网络处理器进入睡眠状态,
以便将网络处理器的执行流停留在特定的位置,以等待下一个数据包的到来。而当待处理的数据包请求到来后,
调度器就将会再次唤醒网络处理器。使网络处理器进入睡眠并停留在特定位置的好处是明显的:当网络处理器
被再次唤醒时,其在执行流中所处的位置将会是确定的,这样大大地降低了运行时软件的设计难度。在设计
运行时的时候可以将这样的位置设定在一次数据包处理的起始地址,这样当网络处理器被唤醒时就不在需要
额外的复位代码来使网络处理器进入起始地址。同时,因为这个过程中网络处理器的执行流不再前进,因此
也就不会产生不必要的访存、计算操作,也就可以起到降低功耗的作用。后续的章节中将会描述睡眠与唤醒
这一机制是如何实现的。

\subsection{功能模块设计}
\label{section:design_module}

\subsubsection{带缓存的Picorv32设计}

在进一步开始访存系统的其他部分设计之前,首先需要解决的是Picorv32的访存性能问题。Picorv32的设计
者出于设计简单小巧的考虑,并没有为Picorv32设计缓存。这显然增大了Picorv32在内存访问上的延时。
为了减小Picorv32在访存上的开销,计划为Picorv32处理器添加数据缓存与指令缓存。

设计一块稳定、可用的缓存会需要不少的时间进行开发,这显然会加大项目的工作量。所幸,当下存在这大量的
开源缓存可供使用。一个名为IOB的项目为本课题提供了可行的解决方案。该项目提供了可行的通用户缓存
解决方案。本课题计划利用IOB项目所提供的缓存实现,对Picorv32进行修改。

在设计中,为Picorv32同时添加了指令缓存与数据缓存。同时,为了避免因数据一致性产生的问题,还应当
留出一条不经由缓存的数据旁路。设立数据旁路主要是出于处理器核与外设之间的数据一致性考虑:如果处理器核
访问外设的通过经过缓存,则处理器核访问的数据就很有可能是缓存是经过缓存的数据,而不是真实的来自外设
数据。这会导致处理器核无法及时访问到来自外设的数据。同样的道理也适用于写内存操作。

\csvgfig{pico_no_cache}{不使用缓存的Picorv32处理器架构设计}{1.0}

\csvgfig{pico_single_cache}{使用单块缓存的Picorv32处理器架构设计}{1.0}

\csvgfig{pico_dual_cache}{使用数据缓存与指令缓存的Picorv32处理器架构设计}{1.0}

\autoref{fig:pico_no_cache}、\autoref{fig:pico_single_cache}与\autoref{fig:pico_dual_cache}分别
展示了三种不同的Picorv32处理器架构设计方案。这三种架构分别对应了不使用缓存、使用统一的一级缓存
以及使用数据缓存、指令缓存以及二级缓存的架构设计方案。

不使用任何缓存的设计来自于Picorv32本身的设计\cite{picorv32}。这样的设计足够简单,面积与频率
都能达到较好水平。同时,因为不存在缓存,也就无需担心缓存所造成的数据一致性问题。但是缺少缓存在
性能上的影响也是非常明显的。后续的章节将会说明这一点。在本课题中,使用本设计作为对照组,与添加
缓存的设计进行性能比较。

使用统一的一级缓存的好处在于结构设计上较为简单,数据通路较为简单。Picorv32在设计上指令访存与
数据访存通过统一的接口访问存储器。因此这样的设计也比较契合Picorv32本身的接口设计。但统一缓存的
缺陷也同样明显。在这样的设计下,指令与数据共用统一块缓存,互相之间发生淘汰的可能性较大,因此可能
会导致缓存出现颠簸,进而造成性能上的损失。

二级缓存设计是经典的通用处理器缓存架构设计方案。在指令缓存与数据缓存分离的同时,使用一块更大的
高速存储器件对数据进行更进一步的缓存处理。显然,这样的方案会带来更多的面积和功耗开销,从而加大
生产、制造成本。在一般的ASIC设计方案中通常会使用片外的高速存储器作为二级缓存,以降低芯片自身
的成本。但这样做的好处也是同样明显的。使用二级缓存可以有效地降低访问缓存时的未命中概率,指令缓存
与数据缓存分离也能一定程度上避免因为在取指操作与去取数据操作相互淘汰而造成的颠簸,避免了无谓的性能
浪费。

\subsubsection{Corundum缓存队列设计}

Corundum使用缓存队列用于缓冲不同部件之间的数据处理速度。Corundum使用的缓存队列是典型的先进先出
(First In First Out,FIFO)结构,使用一组入队接口与出队接口用于连接缓存队列前后的处理模块。
这样的设计显然不能满足本课题的设计要求。为了实现本课题所需要的功能,需要在Corundum缓冲队列的基础
上增加新的端口设计。

新增加的端口主要可以分为两个部分:请求交互接口和访存接口。其中,请求交互接口连接缓存队列与调度器,
用于向调度器发送数据包处理请求,并接收来自调度器的回复信号。请求信号采用如\autoref{tbl:design_req_signal}所示
的格式。设计上使用一对握手信号来平衡缓存端口与调度器之间的数据包处理速度。此外,信号中包含了请求
所对应的数据包在缓存队列中的起始地址。使用起始地址可以唯一确定队列中的任意数据包,同时传递数据包
的起始地址也便于网络处理器后续对数据包进行寻址访存。

\begin{generaltab}{请求信号格式}{tbl:design_req_signal}
  \begin{tabular}{ccc}
    \toprule
    信号名 & 信号位宽 & 信号说明 \\
    \midrule
    req\_valid & 1 & 请求握手信号 \\
    req\_ready & 1 & 请求握手信号 \\
    req\_addr & 32 & 数据包起始地址 \\
    \bottomrule
  \end{tabular}
\end{generaltab}

新增的另一类端口是用于网络处理器访问的AXI Lite总线接口。AXI Lite端口是AXI4端口的简化版本\cite{intro_amba_axi}。使用
AXI Lite总线端口的目的在于为网络处理器提供可用的访存端口。该端口连接到XBAR互联结构,用于网络
处理器的访存操作。

\subsubsection{网络处理器调度机制设计}

\autoref{fig:design_controlpath}中给出了整个系统控制通路的设计。这其中最为关键的部件就是
负责对来自Corundum缓存队列的请求进行调度的网络处理器调度机制。该机制负责将来自缓存队列的请求
进行整理与分发,传递给其所管理的网络处理器进行处理。

调度器应用了一条缓冲队列对来自Corundum缓冲队列的请求进行缓存,以避免因为网络处理器较慢的处理
速度而导致Corundum缓冲队列停滞。请求缓冲队列在设计上与普通的缓冲队列基本上是一致的,但本课题为
其加入了一些旨在提升性能的改进。本课题在传统缓冲队列的基础上,加入了Fall-Through功能。这一功能
允许当队列为空,且入队端口与出队端口同时有效时,入队数据可以通过旁路直接从出队端口流出,而不需要
通过缓存队列内的存储单元。这样的设计可以在时序上取得更好的性能表现。

进入了调度器缓冲队列的数据请求将经由分发单元按序分发到各个网络处理器的控制单元。分发单元(Dispatcher)
是调度器的核心单元。该单元将动态地监测各个网络处理器的状态,并根据网络处理器的当前状态考虑为其
分配待处理的数据包请求。分发单元采用经典的Round Robin算法来实现网络处理器之间的负载均衡,以
确保每个网络处理器的性能都能够得到充分的利用。

控制单元是调度器中负责与网络处理器进行直接交互的单元,也就直接负责对网络处理器进行控制。在调度器
中,每一个网络处理器都对应了一个控制单元。控制单元上包含了一系列的控制寄存器,并通过这些寄存器实现
了对于网络处理器的控制。在设计上,网络处理器可以通过访存操作直接访问控制单元上的控制寄存器,从而
实现接收、应答数据包请求的功能。

\begin{generaltab}{控制单元内存布局}{tbl:design_scheduler_ctrl}
  \begin{tabular}{cccc}
    \toprule
    寄存器名 & 寄存器起始地址 & 寄存器位宽 & 寄存器说明 \\
    \midrule
    DOORBELL & 0x0000 & 32 & 门铃寄存器 \\
    PKT\_ADDR & 0x0004 & 32 & 待处理数据包的起始地址 \\
    \midrule
    NP\_BUSY & 不可访问 & 1 & 网络处理器当前状态 \\
    \bottomrule
  \end{tabular}
\end{generaltab}

当网络处理器当前处于空闲状态时,NP\_BUSY寄存器变为逻辑零,以告知分发单元该网络处理器正处于空闲
状态,等待新的数据包处理请求带来。而当数据包处理请求经由分发单元被分配到网络处理器后,数据包处理
请求中的req\_addr一项就会被写入到PKT\_ADDR寄存器中,同时,NP\_BUSY寄存器被置1,网络处理器被唤醒,
从而开始工作。通过运行时软件,网络处理器将会从PKT\_ADDR寄存器上读取到对应的数据包的起始地址,从而
开始访问Corundum缓存队列,读取数据包数据并开始进行处理。当处理完成后,运行时控制网络处理器访问
作为门铃寄存器的DOORBELL寄存器,告知控制单元当前数据包已处理完成。控制单元接收到对门铃寄存器的
访存请求后,重新将NP\_BUSY寄存器置0,使网络处理器进入休眠,从而完成了一次网络数据包处理。

\subsection{本章小结}

本章主要讨论了本课题关于松散RISC-V多核片外访存系统的整体设计。首先,本章重新梳理了本课题所面对
的项目需求,并结合项目需求,对本课题的研究内容及其可行性进行了阐述。然后,本章给出了项目的整体
架构设计,并对设计思路进行了描述。进一步的,本章对如缓冲队列、请求调度、访存架构设计等关键设计要点
作出了详细的描述。通过上述描述,本章给出了本课题设计的数据通路与控制通路设计。在下一章中,将对
这些技术要点的具体实现作出更为细节的描述,并说明在实现过程中所需要着重考虑的技术要点与技术难点。

\section{松散RISC-V多核访存架构实现}

上一章节具体描述了本课题对于松散RISC-V多核访存架构的总体设计。但这样的设计仍然停留在行为级别的
描述与建模上,距离具体的技术落地还存在着一定的距离。本章将会从具体实现的角度对所设计的模块进行
描述,并着眼于解决实现过程中所需要解决的技术要点,以及所遇到的技术难点。

\subsection{支持内存访问的缓存队列实现}

对Corundum缓存队列进行改造是本课题所面临的第一个技术关键点。本课题需要通过相关设计,使得作为
处理单元的网络处理器能够通过访存操作高效地访问存储于缓存队列上的数据。这样的要求使得缓存队列需要
同时具有随机访问内存能够通过地址进行数据访问的特点,也同时支持传统缓冲队列先进先出的数据行为。

\subsubsection{缓存队列实现的技术思路}

传统的缓存队列通常基于寄存器或是单口随机访问存储器实现。在设计中,寄存器组或是单口RAM为缓存队列
提供了存储单元,用于存放进入队列的数据单元。同时,缓存队列会使用两个寄存器作为当前队列的队首与
队尾指针。队首指针寄存器指向了缓存队列最靠前的数据单元在存储器上的地址,而队尾指针则指向了最靠后
数据单元在存储器上的地址。当进行入队与出队操作时,缓存队列可以通过队首指针寄存器与队尾指针寄存器
中所保存的地址对存储单元(寄存器或是RAM,这两种方案在时序上略有不同)进行访存,从而完成入队与出队
的操作。

从这样的设计中可以看出,缓存队列内部存在着一块连续的内存可供访问。因此,只要使外部访存操作能够
正确访问缓存队列内部的内存,就可以实现内存访问。而要实现相关功能,则需要为内存增加新的访存端口。

\subsubsection{访存操作与出入队操作的写地址冲突问题}

无论是使用寄存器方案还是随机访问存储器,都存在着可以同时支持两组访存端口的解决方案。然而,采用
两组访存端口也必然会引入一个问题:如何解决因两组访存端口同时写同一个地址所面对的冲突?得益于
网络数据包处理的特殊应用场景,本课题中该问题得以被避开。尽管同时存在着用于数据包出入队的访存端口
以及用于网络处理器访问的访存端口,这两个端口在进行访存操作时地址却不会发生冲突。这是因为网络处理器
所访问的总是已经进入了缓存队列的数据包,而队尾指针指向的则永远是下一个待进入的数据包地址,因此这
两者之间也就不会发生写地址冲突。

\subsection{数据包请求调度器的实现}

\autoref{fig:design_controlpath}给出了系统的控制通路设计。而这其中最为重要的部件就是负责
对数据包请求进行调度与分发的调度器。本节将对数据包请求调度器的具体实现进行介绍,结合具体设计说明
数据包请求调度器是如何完成调度,实现与缓冲队列以及网络处理器之间的交互的。

\subsubsection{数据包请求通信机制的实现}

数据包请求被用于在缓存队列与调度器之间传递信息。当数据包进入队列时,缓存队列会像调度器发送数据包
处理请求,以便调度器唤醒网络处理器,开始对数据包进行处理;而当网络处理器处理完成后,调度器会向
缓存队列发送数据包请求回复,告知缓存队列该数据包已经处理完成,可以离开队列。

\autoref{tbl:design_req_signal}展示了一组数据包请求信号的信号格式。请求信号包含了数据包的
在缓存队列上的起始地址,以及一对握手信号。当数据包请求信号正确地完成握手时,缓存队列就成功地完成
了一次数据包请求的传输。数据包请求中的起始地址则可以通过缓冲队列的队首指针寄存器得到。

\csvgfig{impl_req_fifoside}{缓存队列至调度器方向数据包请求信号通路}{1.0}

\autoref{fig:impl_req_fifoside}展示了缓存队列到调度器方向的数据包请求信号通路。为了能够更加
清晰地展示核心逻辑,图中省略了实现中不相关逻辑信号,并对逻辑做了简化。s\_axis\_last与s\_axis\_ready
是数据包入队端口的AXI Stream总线。其中,s\_axis\_last信号用于表明当前数据是总线传输的最后一拍\cite{arm2011axi},
Corundum使用该信号来标明数据包传输的结束。因此,收到s\_axis\_last信号就意味着当前数据包上的
数据已准备就绪,可以向调度器发送数据包处理请求。所以\autoref{fig:impl_req_fifoside}将
s\_axis\_last信号直接接到了req\_valid信号端口。req\_ready作为握手信号的另一部分,其由调度器
提供,用于表示调度器当前是否可以接收新的数据包处理请求。

pkt\_first寄存器信号用于标注当前拍是否表示全新的数据包传输。pkt\_first寄存器的维护同样与
s\_axis\_last信号有关。在Corundum使用s\_axis\_last信号表示上一个数据包传输结束的同时,该信号
同样也标明了新的数据包传输的开始。因此,在设计上选择将s\_axis\_last信号直接接入到pkt\_first
寄存器的数据端,并在每一次握手成功时对该寄存器进行更新。

first\_addr寄存器信号则用于保存当前数据包的起始地址。该寄存器的数据输入端直接与缓冲队列的队首
指针寄存器相连。当所接收到的数据为数据包的第一拍时,就可将队首指针寄存器fifo\_head的值更新到
first\_addr寄存器中,并在当前数据包接收完成后作为数据包请求信号发送给调度器。

当网络处理器完成了数据包的处理后,调度器会向缓存队列发送数据包请求回复。数据包请求回复信号的组成
与数据包请求信号的组成是一致的。\autoref{fig:impl_req_schedside}显示了简化后的实现逻辑。
因为数据包请求回复信号的处理过程有软件参与,所以数据包请求回复信号的实现逻辑会相对更为简单。

\csvgfig{impl_req_schedside}{调度器至缓存队列方向数据包请求回复信号通路}{1.0}

\autoref{fig:impl_req_schedside}展示了调度器至缓存队列方向的信号通路。当网络处理器完成了
对数据包的处理后,其会访问DOORBELL寄存器来告知调度器数据包已经处理完成。此时DOORBELL寄存器
被置1。考虑到调度器会同时管理多个网络处理器,因此使用Round Robin算法对多个控制器中的DOORBELL
信号进行调度(图中“RR”部分)生成选择信号Sel。通过选择信号Sel,调度器借由多路选择器在多个就绪的
控制器中选择一个,向缓存队列发送数据包请求回复信号。当数据包请求回复信号握手成功时,DOORBELL寄存器
便会被清零,等待下一个待处理的数据包到来。

\subsubsection{网络处理器控制逻辑的实现}

数据包请求调度器的另一个重要功能是实现对网络处理器的控制逻辑。在上一章中提到,调度器使用睡眠与唤醒
机制实现对网络处理器的控制。本节将对这一机制的实现原理进行描述。

\subsubsubsection{时钟门控技术}

网络处理器睡眠与唤醒的基本原理是被称为时钟门控(Clock Gating)的技术。这是一种常见于低功耗设计
的时序逻辑控制手段。该方案的要点在于通过控制逻辑,根据需要暂停或开启被控制模块的时钟信号,从而实现
对被控制模块的控制。

现代片上系统中有着大量的时序逻辑,而这些时序逻辑则均需要时钟进行驱动。因此片上系统中往往会存在
复杂的时钟通路。因为时钟通路由于其特性,总是呈现出树状结构,因此片上系统的时钟通路又被称为时钟树。
因为所有的时序逻辑均依赖时钟树进行驱动,因此时钟树是片上系统中最为重要的信号通路。良好的片上系统
设计对时钟树稳定性的要求极高。

而时钟门控技术则需要对被控制模块的时钟树进行修改。因此对于时钟门控电路的设计必须极为慎重。本课题
所使用的时钟门控电路主要来自于PsPIN项目\cite{di2020pspin}。PsPIN项目使用了一种简单而有效的
组合逻辑电路来实现时钟门控。通过锁存电路,该设计避免了门控过程中可能产生的毛刺,并保证了门控解除
时时钟信号能稳定地产生上升沿。

\csvgfig{impl_ctrl_clkgating}{时钟门控电路原理图}{0.6}

\subsubsubsection{控制寄存器设计}

\autoref{tbl:design_scheduler_ctrl}中给出了网络处理器控制逻辑中相关的控制寄存器。其中,
NP\_BUSY寄存器直接连接到时钟门控电路作为使能端对网络处理器进行时钟门控。而DOORBELL寄存器则作为
作为“门铃寄存器”供网络处理器访问。当网络处理器完成处理时,其会通过访存操作访问DOORBELL寄存器,
而当DOORBELL寄存器被访问后,其通过控制逻辑关闭NP\_BUSY寄存器,停止网络处理器的输入时钟,使
网络处理器进入睡眠。同时,DOORBELL寄存器还被用于标记数据包是否已经处理完成。\autoref{fig:impl_req_schedside}中
的逻辑便是使用了DOORBELL寄存器来判断数据包是否已经处理完成。

\subsubsubsection{网络处理器运行时}

运行时(Runtime)是用于在计算机上提供基本的执行环境的代码。其负责与底层软硬件交互,并向上层软件
提供一套可用的编程接口抽象。在本课题的应用内,因为网络处理器应用场景的特殊性,其需要独特的运行时
来保证网络处理器可以正确地完成相关功能。在本课题中,网络处理器运行时将负责完成如下功能。

\begin{enumerate}
  \item 设立内存栈,建立基本的C语言运行环境
  \item 从网络处理器控制逻辑中获取数据包首地址
  \item 建立栈帧,进入数据包处理逻辑
  \item 在数据包处理完成后访问网络处理器控制逻辑,标记数据包已处理完成
\end{enumerate}

网络处理器通过访存操作实现对网络处理器控制逻辑的访问。\autoref{tbl:design_scheduler_ctrl}展示
了控制逻辑上控制寄存器的相关地址。为了能够为上层应用的编写提供较好的抽象,网络处理器运行时应当完成
对相关控制寄存器的访问。基于对控制寄存器的访问结果,建立用于数据包处理逻辑的栈帧,从而实现底层逻辑
对数据包处理逻辑的抽象。

尽管网络处理器控制逻辑提供了用于时钟门控的相关逻辑,但时钟门控逻辑仍然需要知晓何时对网络处理器进行
控制。在上文已经提到,当网络处理器完成了对数据包的处理后,其将访问DOORBELL寄存器告知控制逻辑,
数据包已经处理完成。因此,网络处理器运行时应当在合适的时机访问位于网络处理器控制逻辑上的DOORBELL
寄存器。

当最初的初始化完成后,网络处理器便总是处于“唤醒-处理数据包-睡眠”的循环之中。因此,网络处理器运行时
应当采用循环设计,以便网络处理器总是能够以相同的状态实现对网络数据包的处理。同时,考虑到网络处理器
在数据包处理完成后进入睡眠,而在新的数据包就绪时被唤醒。因此对DOORBELL的访问应当介于两次数据包
处理之间。

\csvgfig{impl_runtime}{网络处理器运行时流程图}{0.8}

\subsubsection{轮询调度算法与数据包请求调度}

Round Robin算法又称轮询调度算法,是一种通过轮询访问进行请求调度的算法。该算法通过轮询的方式,
实现请求的分发与调度工作。该算法的优势在于其实现简单,且能够提供一定程度上的负载均衡,因此较为
适合被应用于本课题的应用场景。

本课题使用掩码和状态码实现Round Robin算法。网络处理器当前的状态构成了信号状态码state。同时,
使用掩码mask作为选择信号。掩码mask表示在当前选择中有效的选择范围。结合状态码state与掩码mask,
可以建立如下的状态转移公式

\begin{equation}
state_{masked} = state \land mask
\end{equation}
\begin{equation}
mask_{next} = \sim (state_{masked} \bigotimes (state_{masked} - 1))
\end{equation}

同时,结合state有选择信号sel

\begin{equation}
sel = state_{masked} \land \sim (state_{masked}-1)
\end{equation}

为了便于说明该机制的工作方式,不妨假设一段情景。假设现有4个网络处理器等待调度,则这些网络处理器的
状态构成了4位的信号状态码。同时,使用4位的掩码对状态码进行筛选。\autoref{tbl:impl_roundrobin}展示了
经过四轮调度后状态码与掩码的状态。

\begin{generaltab}{使用轮询调度算法进行调度}{tbl:impl_roundrobin}
  \begin{tabular}{c|cccc}
    \toprule
    调度次数 & $state$ & $mask$ & $state_{masked}$ & $sel$ \\
    \midrule
    1 & 4'b1011 & 4'b1111 & 4'b1011 & 4'b0001 \\
    2 & 4'b1010 & 4'b1110 & 4'b1010 & 4'b0010 \\
    3 & 4'b1000 & 4'b1100 & 4'b1000 & 4'b1000 \\
    4 & 4'b0000 & 4'b0000 & 4'b0000 & 4'b0000 \\
    \bottomrule
  \end{tabular}
\end{generaltab}

注意到,经过四轮调度后,此时掩码变为了$4'b0000$,无法再被继续用于调度。因此,在以上逻辑计算的
基础上,可以加上一段选择判断电路,使得当$mask_{next}$等于0时,将$mask$的值替换为全1。这样就
允许电路逻辑在完成一轮调度后再进行下一轮调度了。

基于上述原理,可以很容易地设计出数字电路用于实现轮询调度算法。\autoref{fig:impl_roundrobin}展示了
在本课题中这一电路的大致逻辑。通过这样的设计,来自网络处理器的状态被编码为统一的状态信号,经过
轮询调度器,产生选择信号。选择信号将会被用于多路选择器,以便从多个网络处理器中选择出一个用于响应
网络数据包请求。

\csvgfig{impl_roundrobin}{轮询调度算法电路示意图}{0.9}

\subsection{带缓存的Picorv32处理器实现}

作为真正负责进行网络数据包处理的网络处理器而言,Picorv32在架构中扮演着举足轻重的角色。因此,
Picorv32在架构中的性能表现也就变得尤为重要。然而,处于面积与时序上的优化考虑,Picorv32牺牲了
不少的性能\cite{picorv32}。这使得其在性能上的表现显得难以让人满意。与大多数的处理器不同,Picorv32
并未采用流水线设计,相反其采用了更为简单的多周期处理器设计。这使得Picorv32的CPI——即是是在理想
情况下——也往往能达到3个周期及以上。

与此同时,Picorv32所采用的访存系统也同样存在着较大的优化空间。出于设计的简单考虑,Picorv32处理器
并未使用缓存设计,而是使用了被称为Picorv32 Native Memory Bus\cite{picorv32}的自定义访存
总线用于基本的访存操作。在此基础上,为了方便集成,Picorv32还提供了支持AXI Lite总线的顶层实现。

\begin{generaltab}{Picorv32自定义访存总线\cite{picorv32}}{tbl:impl_picorv32_nativebus}
  \begin{tabular}{cccc}
    \toprule
    信号名称 & 信号位宽 & 信号方向 & 信号说明 \\
    \midrule
    mem\_valid & 1 & output & 握手信号 \\
    mem\_instr & 1 & output & 是否为取指操作 \\
    mem\_ready & 1 & input & 握手信号 \\
    mem\_addr & 32 & output & 访存地址信号 \\
    mem\_wdata & 32 & output & 写数据信号 \\
    mem\_wstrb & 4 & output & 写掩码信号 \\
    mem\_rdata & 32 & input & 读数据信号 \\
    \bottomrule
  \end{tabular}
\end{generaltab}

缺少缓存对于Picorv32的访存性能的影响是非常严重的。考虑到现代访存架构中对于片外存储设备的访问往往
速度缓慢,需要大量的周期完成,因此在缺少缓存的情况下,Picorv32不得不花费上百甚至上千个周期来完成!
这对于追求速度的网络数据包处理应用而言显然是无法接受的。因此,为了得到较好的性能表现,有必要在
Picorv32的基础上增加缓存系统。

\subsubsection{缓存架构的实现}

得益于现有的开源项目,本课题并不需要从零开始实现一套缓存系统。名为IOB-Cache的开源工作\cite{roque2021iob}为
本课题提供了可行的开源实现。IOB-Cache是目前不多的开源缓存项目之一,其提供了良好的可配置性,与较好
的访存性能。同时,IOB-Cache还提供了性能优异的采用“写穿”策略的缓存与控制逻辑设计。这样的设计使得
IOB-Cache可以提供开销接近于一个周期的近乎理想的写内存操作。于此同时,IOB-Cache还提供了现有的
集成SoC环境可供参考,在该集成设计中,CPU的指令平均周期数(CPI)指标最好能够达到1.055,非常接近
于理想情况下的CPU实现。

同时,IOB-Cache在接口上与Picorv32采用了几乎一致的接口。尽管其接口与Picorv32的访存接口在时序
上存在些许差别,但在实现上仍然可以较为轻松地完成Picorv32与IOB-Cache的连接。在\autoref{section:design_module}中,
由\autoref{fig:pico_no_cache}、\autoref{fig:pico_single_cache}以及\autoref{fig:pico_dual_cache}分别
给出了三种不同的缓存架构设计设计思路,同时也讨论了采用不同缓存架构所带来的设计上的好处与坏处。综合
前文的考量,本课题认为单块缓存的设计是最为符合本课题需求的:该设计预期将有着相对较好的访存性能表现,
于此同时,该实现相对较为简单,对时序与面积的影响也相对较小。

\autoref{fig:pico_single_cache}已经给出了较为具体的架构设计,因此缓存架构在实现上的主要工作
为连接与集成所用到的各个IP。下文将逐步介绍各个IP在架构中所扮演的角色,并对实现思路进行解释。

IOB-Cache与Picorv32均使用了被称为Native Memory Bus的自定义总线作为其基础的访存总线方案。
这使得在实现中可以容易地将IOB-Cache的前端与Picorv32通过总定义总线连接起来。\autoref{tbl:impl_picorv32_nativebus}展示了
该总线的信号构成。通过一对握手信号,该总线可以通过一次握手中完成读数据操作或是写数据操作。

简单地自定义总线也为多路选择器的实现提供了便利。为了实现不经过缓存的旁路功能,需要对由处理器核引出
的访存总线进行解码与分发。出于设计简单的考虑,在实现中选择使用地址总线的最高位作为是否经过缓存的
判断信号。当地址总线最高位为1时,访存操作将通过缓存;反之则经由旁路。\autoref{tbl:impl_picorv32_nativebus}中的UseCache信号
也由此产生。

cache\_axi是IOB-Cache所提供的一种实现。该实现选择使用AXI总线作为缓存后端,连接外部的存储系统。
但在本课题中,主要采用AXI Lite总线作为访存系统的通用总线。因此,对于IOB-Cache的缓存后端而言,
需要使用AXI到AXI Lite的总线桥对总线信号进行转换,以便能够正常地连接到外部的存储设备。同理,旁路
上则使用了Picorv32项目自带的AXI Lite总线转接模块进行连接。

缓存通路与旁路上的AXI Lite总线信号最终将通过多路选择器,由UseCache信号进行选择,经处理后与外部
的存储系统相连接。

\subsection{片外存储设备的访问实现}

片外存储设备的访问将主要通过AMBA总线完成。Alveo U280数据中心加速卡提供了丰富的片外存储资源可供
使用,包括总容量达32GB的DDR存储器,以及8GB大小的HBM2高带宽存储器\cite{alveo2021datasheet}。通常,
片外存储设备通过引脚与片内的存储控制器设计相连。片内逻辑访问片外存储设备可以通过访问位于片内的
存储控制器完成。Vivado设计工具组针对Alveo U280提供了完善的片内控制器IP可供使用。片内控制器IP
通常支持如AXI总线等常见的AMBA总线协议。因此,本课题可以容易地使用AXI Lite总线协议实现对片外存储
设备进行访问。

\subsection{本章小节}

本章主要对松散RISC-V多核访存架构的实现细节进行了描述。沿着数据包在数据通路上传输的顺序,本章首先
对支持内存访问的缓存队列实现进行了介绍,说明了本课题如何在Corudum缓存队列的基础上添加内存访问支持。
然后,作为设计的重点,本章着重描述了数据包请求调度器与带缓存的Picorv32处理器的实现。通过这些描述,
本章详细地介绍了本设计如何通过控制通路与访存架构对网络数据包进行处理。最后,本章对Alveo U280数据
中心加速卡上访问片外存储设备的方式进行了介绍。

\section{性能测试与分析}

本章将对本课题目前的开发进度进行介绍,说明课题目前所取得的进展。同时,本章将介绍本课题目前所做的
一些实验性测试,对实验数据进行分析,并解读实验数据对于本课题的意义。

\subsection{测试环境}

本课题的所有开发与测试工作均在Linux操作系统下进行。本课题的开发与测试分别在本地与远程服务器上进行。
视开发测试环境不同,本课题所使用的Linux发行版亦有所不同。在本地环境下,所使用的Linux发行版为
Fedora 34 x86\_64版本,所使用的内核版本则为5.13.12-200.fc34.x86\_64。而在远程的服务器上,
使用的Linux发行版则为Ubuntu 18.04,内核版本为5.4.0-80-generic。

\subsection{带缓存的Picorv32处理器架构的仿真测试}

对于带缓存的Picorv32的仿真测试使用Verilator高性能仿真工具进行仿真测试。Verilator作为仿真工具,
将Verilog硬件描述语言转换为对应的C++语言代码,通过GCC进行编译得到用于仿真的可执行程序。同时,
Verilator高性能仿真工具还能通过函数调用生成波形图,便于开发过程中的调试与仿真。

\subsubsection{测试框架的编写}

对于带缓存的Picorv32处理器框架的仿真测试与验证是在被称为Testbench的测试框架中进行的。Testbench
是一种不可综合的仿真测试框架,通常会使用一些不可综合的仿真语法,以便与实现一些便于调试但在实际的
开发板上难以实现的功能,如延时、打印或读取文件等。

对于带缓存的Picorv32处理器架构而言,测试框架主要为其提供了一片可通过AXI Lite总线进行访存的仿真
内存模型。该模型能够提供一定大小的内存区用于测试程序的存储与运行。同时,该模型还提供了便于调试的
串口输出功能,以便于测试程序对运行情况进行输出,便于调试。处理器核可以通过访问特定的内存地址实现
来进行输出与调试。在设计上,指定0x1000\_0000为串口输出的地址,处理器核通过向该地址写入一个字节
的数据来实现单个字节的串口输出。当测试程序执行结束时,处理器核向0x2000\_0000处写入特定的验证
数据,以表示测试程序的正常结束。

除此之外,测试程序还提供了如下的仿真测试功能。

\begin{enumerate}
  \item 对带缓存的Picorv32处理器核架构进行配置与例化,并将其与测试框架相连接;
  \item 对仿真内存模型进行例化;
  \item 将用于测试的测试程序载入到仿真内存模型中;
  \item 为仿真测试系统提供时钟与复位信号;
  \item 监测处理器核对测试程序的执行情况,在测试程序正确执行完成后结束仿真测试;
  \item 根据配置,利用\$dumpfile系统函数对调试过程中的模块信号进行记录,生成并输出波形。
\end{enumerate}

\subsubsection{仿真测试结果}

在本课题的测试中,主要比较了带缓存的Picorv32处理器核架构与不带缓存的Picorv32处理器核架构在执行
特定测试程序时的性能差异,以衡量带缓存的Picorv32处理器核设计在性能上是否优于不带缓存的Picorv32
处理器核设计。同时,通过调节仿真测试框架中仿真内存模型的访存时延,本课题将能得到两种设计在不同的
外部访存速度下的性能差异。

\begin{generaltab}{不同访存延时下两种设计的性能差异}{tbl:verif_pico_delay}
  \begin{tabular}{c|cc}
    \toprule
    访存延时(周期) & CPI(带缓存) & CPI(不带缓存) \\
    \midrule
    5 & 6.82 & 13.87 \\
    10 & 7.46 & 21.49 \\
    15 & 8.20 & 29.84 \\
    20 & 9.04 & 38.93 \\
    25 & 9.94 & 48.92 \\
    \bottomrule
  \end{tabular}
\end{generaltab}

可以看到,随着访存延时的增加,带缓存的架构设计与不带缓存的架构设计在测试中所得到的性能表现均呈现出
了下降态势。这是符合实验预期的。但同时从数据上也可以看出,带缓存的架构设计的CPI随着访存延时的增加
而增加的增幅要远远小于不带缓存的架构设计。随着访存延时的增大,这两种架构设计之间的性能差异也就越来
越大。可以看到,当访存延时达到25个周期时,带缓存的设计与不带缓存的设计之间的性能差异可以达到4倍
以上。

因此,结合\autoref{tbl:verif_pico_delay}所给出的数据,为Picorv32增加缓存能够显著地提升其
在访存系统中的性能表现。因而也更适合被用于智能网卡上的网络报文处理应用。

\subsection{基于Corundum的访存架构的测试与验证}

本课题完成了对于Corudum缓存队列的相关修改,并在此基础上进行了测试。出于测试的简单性考虑,同时也
受制于现有的项目进度,目前对于松散RISC-V多核架构的仿真测试主要是针对缓存队列的测试。本课题完成了
对于缓存队列的相关修改,并接入了单个的Picorv32处理器核作为网络处理器对数据包进行修改。在已完成
工作的基础上,本课题对所实现的基于Corundum缓存队列的访存架构进行了仿真测试与上板验证。

\subsubsection{使用Cocotb仿真测试框架进行仿真验证}

Corundum项目提供了基于Cocotb测试框架的仿真测试用例,用于对项目中各个模块单独进行仿真测试,验证
正确性。Cocotb是基于Python语言的快速仿真测试框架,可以用于快速地生成数字电路仿真测试用例\cite{cocotb_doc}。
结合已有的仿真测试框架,本课题对使用了支持内存访问的缓存队列的Corundum架构进行了仿真验证,并得到
了如\autoref{fig:verif_cocotb}所示的仿真结果。

\cpngfig{verif_cocotb}{使用Cocotb对基于Corundum的访存架构进行测试}{0.8}

对Corundum所进行的修改能够正确地通过Corundum项目所设立的仿真测试,因此可以认为本课题所作的修改
在实现上是正确且符合要求的。

\subsubsection{基于Xilinx Alveo U280的上板验证}

在完成了基于软件的仿真测试后,选择对架构进行FPGA上板验证。本课题上板验证的内容主要包括了基础功能
测试与性能测试。基础功能验证主要测试修改后的Corundum,作为集成了RISC-V处理器核的可编程智能网卡,
在部署上FPGA后,能够进行正常地网络通信;而性能测试则是对于这一过程的如带宽、延时等性能参数的测量。

为了将设计部署上FPGA,需要将Xilinx Alveo U280板卡插入到带有PCIe插槽的服务器主机上,并通过远程
访问服务器的方式进行部署与调试。同时,为了测试Xilinx Alveo U280的网络功能,使用100GbE的网络
线缆连接Xilinx Alveo U280板卡与部署在另一主机上的100GbE网卡,并配置好网络IP。完成硬件连接后,
使用Vivado开发工具组载入项目,生成比特流(Bitstream)并烧录到Xilinx Alveo U280上,即可完成
部署。

\subsubsubsection{基础功能测试}

对基础功能的测试将通过Linux系统自带的ping命令完成。在部署了U280板卡的主机上,使用ping命令进行
与另一主机的连通测试。

\cpngfig{verif_u280_connect}{基于U280板卡的基础功能测试}{0.8}

可以看到,使用ping命令能够正确地建立与物理上相连接的另一主机之间的网络连接,因此可以认为部署的
基础功能测试是成功的。

\subsubsubsection{性能测试与分析}

对于性能的测试则使用iperf工具完成。iperf工具是一种用于主动测量IP网络最大带宽的开源工具,支持多种
参数配置,并能够使用多种协议(如TCP、UDP等等)。在本测试中,对经过本课题修改后的实现进行了测试,
以分析本课题所设计访存架构对于Corundum网络接口控制器的性能影响。

\autoref{fig:verif_u280_iperf}展示了本课题的设计在性能测试中的输出。可以看到,经过本课题修改
后的带宽大概为2.8Gbps。而同样是在本课题的测试中,Corundum能够达到94Gbps的带宽。

\cpngfig{verif_u280_iperf}{基于U280板卡的性能测试}{0.9}

对比之下这样的结果显然并不算好。由此可见目前的实现仍然存在诸多可供完善的部分首先,目前的测试基于
单个Picorv32处理器核进行,侧重于验证基于Corundum的访存架构修改的可行性,而非最终的性能测试;
其次,在测试中Picorv32处理器核通过旁路访问缓存队列读取其上的网络报文,这一过程通过多个总线
结构,因此访存开销较大,影响了处理器核的性能表现。

\subsection{本章小结}

在本章中,主要介绍了本课题目前所完成工作内容,并对目前所做的各种测试工作进行了描述。本章首先介绍了
本课题关于缓存架构对性能影响的测试。在这个测试中,通过比较带缓存的Picorv32架构与不带缓存的Picorv32架构,
本课题分析了缓存对于Picorv32架构的性能影响。然后,本章总结了目前基于Corundum所做的松散RISC-V多核
架构设计工作,并进行了上板验证与性能测试。本章还对性能测试的结果进行了分析,并给出了可能对性能造成
影响的各类技术因素。

\section{总结与展望}

本章主要对前文的内容进行总结,介绍当前本课题所进行的研究工作情况。结合课题的设计与目前的工作进展,
对课题的研究情况作出结论与思考。本章还将讨论当前课题在设计中所考虑的制约因素,以及对于实施本课题的
成本估算。

\subsection{主要内容与总结}

本课题主要探究了松散RISC-V多核访存架构的实现方式。基于Picorv32处理器核,本课题设计并部分构建了
基于RISC-V指令集的采用松散一致性存储模型的多核架构。作为本课题的一部分,本课题还探究了缓存系统
对于处理器核在访存架构中的性能影响。通过结合开源项目IOB-Cache\cite{roque2021iob},本课题为
Picorv32处理器核搭建了缓存架构,并通过实验探究了不同缓存延时对于带缓存的架构与不带缓存的架构的
处理器核执行性能的影响。

本课题同时探究了松散RISC-V多核架构在智能网卡上的实现。结合开源的网络接口控制器项目Corundum\cite{forencich2020corundum},本课题
研究了Corundum的数据通路设计,并讨论了对Corundum数据通路进行修改,以集成多处理器核架构的技术方案。
本课题对Corundum数据通路上的缓存队列进行了修改,为其添加了内存访问功能,并支持AXI Lite总线协议
访问。同时,基于Corundum上数据包处理的特性,本课题设计并部分实现了面向多核系统的数据包请求调度
机制。调度机制使用轮询调度算法对数据通路上的数据包处理请求进行调度,并通过时钟门控技术实现了处理器
核的睡眠与唤醒控制。

本课题对修改后的Corundum架构进行了基于Cocotb\cite{cocotb_doc}与Verilator\cite{verilator_website}高性能仿真工具的
的软件仿真测试。并在远程服务器上对设计进行了基于FPGA的上板测试。针对上板测试,本课题使用Linux下的
网络工具对所部署的代码进行了连通性测试与性能测试。结合性能测试所得到的数据,分析了当前实现中可能
造成性能影响的各类原因。

受制于项目难度、工作时间等因素,本课题仍然存在着许多尚不完善的设计与实现,部分实现也未得到充分的
测试与验证。\autoref{section:conclusion_future}中将会对这些因素进行描述。

\subsection{设计中考虑的制约因素}

% 简要说明在方案设计和实现、技术选择或平台选择等方面,为减少你所开发或设计的系统实施后可能存在的
% 对社会、健康、安全、法律、文化及环境等方面带来的不利影响,你在设计中采取的相关措施。

\subsubsection{知识产权方面的考量}

知识产权问题是在进行平台选择与技术方案制定时需要首先考虑的问题。在硬件设计领域,不少的商业硬件知识
产权核(Intellectual Property,IP)在使用前都往往需要支付昂贵的使用费用。而使用盗版有面临着
严重的法律风险。出于学术研究的目的,本课题在选择平台时选择了使用开源证书的开源项目作为本课题的技术
平台。开源项目天生的包容性与开放性可以避免因知识产权纠纷所带来的法律风险。

本课题中所使用的如RISC-V指令集、Corundum开源网络接口控制器、Picorv32处理器核、Verilator高性能
仿真器等项目均属于开源项目,在遵守对应的开源协议的前提下,可以在本课题中自由使用。因而避免了潜在的
法律风险。

\subsubsection{技术生态方面的考量}

技术生态是本课题在开展时所考虑的另一重要因素。好的技术技术生态往往能够保证开源项目更加可靠,并在
使用该项目时能够更容易地获得技术支持。同时,在良好的技术生态下进行开发也意味着项目更容易受到其他
技术人员的关注。

因此,本课题在选择技术平台时同样对开源项目的技术生态进行了相关的考察。如Verilator、Picorv32等
开源项目已经得到了相关研究领域的广泛使用,因而具有良好的技术生态。在基于这些项目进行开发时,曾多次
收到来自开发者社区的技术帮助。

\subsection{成本估算}

% 主要概算一下所开发或设计的系统所花费的成本。

本课题所涉及的工作量主要可以分为两个部分:前期的项目调研学习与后期实际的开发调试工作。在前期的调研
过程中,阅读了大量的关于智能网卡、访存架构设计、存储一致性模型等方面的相关研究;同时,还阅读了如
PsPIN、Picorv32、IOB-Cache等开源项目代码。这一部分尽管不涉及具体的开发工作,但却是项目中最为
耗时的部分。这是因为需要详细地阅读论文或是代码,对研究成果产生一定理解,以能够运用研究成果或是
项目代码。本课题大概投入了两个月时间在各种论文、项目的阅读与调研工作上。

后期的工作量则主要来自于项目开发本身。考虑使用Putnam模型对项目的工作量进行分析。本课题预估代码
能达到3000行,计划在半年内完成实现与测试。目前项目开发环境较为完善,有良好的开发、调试手段,因此
技术状态常数取8000。结合Putname模型,可以得到开发所需要的工作量为0.84个人年。

\subsection{未来展望}
\label{section:conclusion_future}

% 对未来的展望,补充一下项目中尚未完成的工作

受制于课题时间有限、个人能力有限、项目整体时间规划等因素,本课题所涉及的项目仍然处于尚未完成的状态,
且目前仍然面临着诸多技术挑战等待克服。作为对下一步工作的规划,未来的工作将围绕如下几点展开:

\begin{enumerate}
  \item 对本课题所设计的多核架构,包括多核下的数据包请求调度机制以及网络处理器控制逻辑等设计,
  继续完成实现,并进行仿真与上板验证;
  \item 现有的缓存队列设计在访存时仍然存在成为性能瓶颈的可能,未来的工作应当考虑对多核控制下的
  缓存队列设计进行优化;
  \item 目前实现网络数据包处理的算法直接使用了开源算法进行,不能充分地利用现有的硬件设计,未来
  应当考虑对算法进行优化。
\end{enumerate}

随着网络带宽的快速增长,而随着这样的提升的,则是计算机网络对于网络接口控制器要求的不断提升。另一
方面,互联网技术地快速发展也使得计算机网络应用正变得日趋复杂,且日新月异。传统的基于固定卸载的
网络控制器无法持续满足日新月异的网络应用需求。而智能网卡凭借着其具有可编程性这一特点,可以灵活地
根据需求更换卸载,因此必将成为计算机网络领域未来的一大发展焦点。

结合网络数据包处理的特殊应用场景,在多核架构下的松散一致性存储模型能够更好地协调智能网卡上网络处理器
之间的存储一致性问题。相信随着技术的发展,这一架构设计思路也将会在智能网卡领域得到更多的应用。


\begin{thankpage}

% TODO: add a thankpage following this order

% 我首先要对我的指导老师邵志远教授表示衷心的感谢。。。

% 然后,我要对实验室里的各位学长学姐表示诚挚的感谢。感谢朱志成、霍振飞两位学长在毕设课题完成期间
% 提供的诸多帮助与指导,还要感谢吴敏康、陈佳杰、黄硕、肖靖宇和李可欣五位学长学姐。。。

% 我要感谢我的家人。。。

% 我还要感谢我长期以来的朋友陆思彤、车春池、何叔恒、刘攀、杨君毅、陈锦、禹露等人。。。

% 此外,我还要感谢计卓1801班的李志高、梁昌洵、陈柏余、金明明等全体班干部成员。。。

% 最后,我要感谢我的母校华中科技大学。。。


\end{thankpage}

\bibliography{main}

\end{document}
