\section{总结与展望}

本章主要对前文的内容进行总结,介绍当前本课题所进行的研究工作情况。结合课题的设计与目前的工作进展,
对课题的研究情况作出结论与思考。本章还将讨论当前课题在设计中所考虑的制约因素,以及对于实施本课题的
成本估算。

\subsection{主要内容与总结}

本课题主要探究了松散RISC-V多核访存架构的实现方式。基于Picorv32处理器核,本课题设计并部分构建了
基于RISC-V指令集的采用松散一致性存储模型的多核架构。作为本课题的一部分,本课题还探究了缓存系统
对于处理器核在访存架构中的性能影响。通过结合开源项目IOB-Cache\cite{roque2021iob},本课题为
Picorv32处理器核搭建了缓存架构,并通过实验探究了不同缓存延时对于带缓存的架构与不带缓存的架构的
处理器核执行性能的影响。

本课题同时探究了松散RISC-V多核架构在智能网卡上的实现。结合开源的网络接口控制器项目Corundum\cite{forencich2020corundum},本课题
研究了Corundum的数据通路设计,并讨论了对Corundum数据通路进行修改,以集成多处理器核架构的技术方案。
本课题对Corundum数据通路上的缓存队列进行了修改,为其添加了内存访问功能,并支持AXI Lite总线协议
访问。同时,基于Corundum上数据包处理的特性,本课题设计并部分实现了面向多核系统的数据包请求调度
机制。调度机制使用轮询调度算法对数据通路上的数据包处理请求进行调度,并通过时钟门控技术实现了处理器
核的睡眠与唤醒控制。

本课题对修改后的Corundum架构进行了基于Cocotb\cite{cocotb_doc}与Verilator\cite{verilator_website}高性能仿真工具的
的软件仿真测试。并在远程服务器上对设计进行了基于FPGA的上板测试。针对上板测试,本课题使用Linux下的
网络工具对所部署的代码进行了连通性测试与性能测试。结合性能测试所得到的数据,分析了当前实现中可能
造成性能影响的各类原因。

\subsection{设计中考虑的制约因素}

% 简要说明在方案设计和实现、技术选择或平台选择等方面,为减少你所开发或设计的系统实施后可能存在的
% 对社会、健康、安全、法律、文化及环境等方面带来的不利影响,你在设计中采取的相关措施。

\subsubsection{知识产权方面的考量}

知识产权问题是在进行平台选择与技术方案制定时需要首先考虑的问题。在硬件设计领域,不少的商业硬件知识
产权核(Intellectual Property,IP)在使用前都往往需要支付昂贵的使用费用。而使用盗版有面临着
严重的法律风险。出于学术研究的目的,本课题在选择平台时选择了使用开源证书的开源项目作为本课题的技术
平台。开源项目天生的包容性与开放性可以避免因知识产权纠纷所带来的法律风险。

本课题中所使用的如RISC-V指令集、Corundum开源网络接口控制器、Picorv32处理器核、Verilator高性能
仿真器等项目均属于开源项目,在遵守对应的开源协议的前提下,可以在本课题中自由使用。因而避免了潜在的
法律风险。

\subsubsection{技术生态方面的考量}

技术生态是本课题在开展时所考虑的另一重要因素。好的技术技术生态往往能够保证开源项目更加可靠,并在
使用该项目时能够更容易地获得技术支持。同时,在良好的技术生态下进行开发也意味着项目更容易受到其他
技术人员的关注。

因此,本课题在选择技术平台时同样对开源项目的技术生态进行了相关的考察。如Verilator、Picorv32等
开源项目已经得到了相关研究领域的广泛使用,因而具有良好的技术生态。在基于这些项目进行开发时,曾多次
收到来自开发者社区的技术帮助。

\subsection{成本估算}

% 主要概算一下所开发或设计的系统所花费的成本。

本课题所设计的工作量主要可以分为两个部分,前期项目调研学习与实际开发调试。在前期的调研过程中,阅读
了大量的关于如智能网卡、访存架构设计、存储一致性模型等方面的相关研究;同时,还阅读了如PsPIN、Picorv32、
IOB-Cache等开源项目代码。这一部分是课题中目前最为消耗时间的部分,因为需要详细地阅读论文或是代码,
对研究成果产生一定理解,并能够熟练地运行研究成果或是项目代码。本课题大约投入了两个月时间在各种论文、
项目的阅读与调研之上。

本课题的另一大工作量来自于项目本身的开发工作。不同于软件开发,硬件逻辑的开发工作往往需要长时间的
验证与调试。在开发过程中,至少有一半以上的开发时间被用于对各类模块的调试工作。本课题目前在开发工作
上大致投入了两周时间。考虑到项目仍在进行,预计将会投入更多的时间以完成本项目。

\subsection{未来展望}

% 对未来的展望,补充一下项目中尚未完成的工作

受制于课题时间有限、个人能力有限、项目整体时间规划等因素,本课题所涉及的项目仍然处于尚未完成的状态,
且目前仍然面临着诸多技术挑战等待克服。作为对下一步工作的规划,未来的工作将围绕如下几点展开:

\begin{enumerate}
  \item 对本课题所设计的多核架构,包括多核下的数据包请求调度机制以及网络处理器控制逻辑等设计,
  继续完成实现,并进行仿真与上板验证;
  \item 现有的缓存队列设计在访存时仍然存在成为性能瓶颈的可能,未来的工作应当考虑对多核控制下的
  缓存队列设计进行优化;
  \item 目前实现网络数据包处理的算法直接使用了开源算法进行,不能充分地利用现有的硬件设计,未来
  应当考虑对算法进行优化。
\end{enumerate}

随着网络带宽的快速增长,而随着这样的提升的,则是计算机网络对于网络接口控制器要求的不断提升。另一
方面,互联网技术地快速发展也使得计算机网络应用正变得日趋复杂,且日新月异。传统的基于固定卸载的
网络控制器无法持续满足日新月异的网络应用需求。而智能网卡凭借着其具有可编程性这一特点,可以灵活地
根据需求更换卸载,因此必将成为计算机网络领域未来的一大发展焦点。

结合网络数据包处理的特殊应用场景,在多核架构下的松散一致性存储模型能够更好地协调智能网卡上网络处理器
之间的存储一致性问题。相信随着技术的发展,这一架构设计思路也将会在智能网卡领域得到更多的应用。
