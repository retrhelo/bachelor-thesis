\section{松散RISC-V多核访存架构设计}

本章主要对本课题的内容——松散RISC-V多核访存架构——的设计进行介绍。本章首先阐述本设计所面对的功能
需求进行阐述。随后,本章将会描述架构的整体设计,以便能展示架构的大致样貌。最后,本章将会详细解释
架构中部分关键部件的设计。

\subsection{功能需求}
\label{section:design_need}

如\autoref{section:background_smartnic}所言,本课题的主要需求来自于智能网卡上的对于数据包
处理的多核架构需要。已经提到,传统应用场景下所使用的较为严格的存储一致性模型并不能很好地利用智能
网卡应用场景下的数据特点,因此会带来潜在的性能损耗。而使用松散存储一致性模型或许能更好地利用这些
数据特点,实现性能上的提升。

同时,智能网卡这类入网计算的应用场景也需要一套不同于传统应用场景下的访存架构,以便能满足对于高效
网络数据包处理的需求。在这样的访存架构中,负责计算工作的网络处理器(Network Processor,NP)
将会需要一套访存架构来访问数据通路上所传输的网络数据包,并对这些数据包进行处理。

\subsection{系统总体设计}
\label{section:design_overview}

\subsubsection{针对网络数据通路的访存设计}

如\autoref{fig:corundum_datapath}所示,Corundum上的数据在与外部网络进行交互时,会通过端口
上的缓冲队列结构,再离开/进入媒体访问控制器。而为了实现对数据通路上数据包的访问,就需要设计实现
处理器核对数据缓冲队列的访存通路设计。

\csvgfig{design_datapath}{针对网络数据通路的访存设计}{0.8}

\autoref{fig:design_datapath}展示了本课题数据通路设计的基本结构。其中,Picorv32构成的处理器
核将作为网络处理器(Network Processor,NP)使用。同时图中还展示了系统中其他可供网络处理器访问
的存储单元。其中,SRAM模块将为网络处理器提供内存用于堆栈逻辑地址空间的数据存储,并用于存放网络处理器
的部分运行时代码;而CTRL模块则属于网络处理器的控制单元,用于控制网络处理器对数据包的处理流程。
后续的章节中将会对这两个模块进行更为细致的描述。

XBAR是一种常见的片上互联结构(Interconnect)。通常,这样的结构被用于连接片上的各个总线单元,
并对总线上的数据传输进行地址译码与分发。在本课题中,使用XBAR结构来连接各个网络处理器、Corudnum
缓存队列以及片外的DDR等存储器件。使用XBAR结构将可以对多个网络处理器所发起的总线传输操作进行协调
与调度,以保证每个网络处理器都能够实现对存储器件的访问。

图中的FIFO即代表了Corundum上的数据包缓冲队列结构。在具备常规缓冲队列所具备的先进先出功能的同时,
缓冲队列还提供了类似于内存的访问方式。缓存队列所提供的一整套接口允许外部的网络处理器通过访存指令
对其进行访问,以便对其上的数据进行操作。缓存队列所提供的接口允许外部的网络处理器能够已访问内存的
方式对其进行访问,从而简化了访存架构中的对缓存队列的访问设计。

DDR代表了外部的双倍数据率同步动态随机存取存储器(Double Data Rate Synchronous Dynamic
Random-Access Memory,DDR SDRAM)。这是一种拥有较高速率的片外存储设备,且往往具有较大的数据
容量。在设计中DDR将用于存储供网络处理器运行的嵌入式软件程序,并同时保存一些占用空间较大的数据结构。
DDR同样通过XBAR互联结构与访存架构中的其他部件相连,以确保每一个网络处理器都能顺利地访问存储在
DDR上的相关数据。

\subsubsection{系统控制通路设计}

控制通路主要用于控制访存架构中的网络处理器的行为。在设计上会使用调度器对所有的网络处理器进行控制,
同时调度器也会负责将缓存队列结构上的数据包分发给各个处理器核用于处理。

\csvgfig{design_controlpath}{系统控制通路设计}{0.9}

一方面,调度器与Corundum的缓冲队列相连。当缓冲队列收到新的数据包时,其就会想调度器发送包含数据包
相关信息的请求(Request)。而调度器收到请求后将会该请求传递给分发器(Dispatcher),再由分发器
将数据包发送给对应的网络处理器进行处理。

另一方面,调度器通过与网络处理器的连接来实现对网络处理器的控制。调度器通过睡眠与唤醒机制,来实现
所需要的控制行为。如果当前不存在数据包共网络处理器处理,那么调度器就会使网络处理器进入睡眠状态,
以便将网络处理器的执行流停留在特定的位置,以等待下一个数据包的到来。而当待处理的数据包请求到来后,
调度器就将会再次唤醒网络处理器。使网络处理器进入睡眠并停留在特定位置的好处是明显的:当网络处理器
被再次唤醒时,其在执行流中所处的位置将会是确定的,这样大大地降低了运行时软件的设计难度。在设计
运行时的时候可以将这样的位置设定在一次数据包处理的起始地址,这样当网络处理器被唤醒时就不在需要
额外的复位代码来使网络处理器进入起始地址。同时,因为这个过程中网络处理器的执行流不再前进,因此
也就不会产生不必要的访存、计算操作,也就可以起到降低功耗的作用。后续的章节中将会描述睡眠与唤醒
这一机制是如何实现的。

\subsection{功能模块设计}
\label{section:design_module}

\subsubsection{带缓存的Picorv32设计}

在进一步开始访存系统的其他部分设计之前,首先需要解决的是Picorv32的访存性能问题。Picorv32的设计
者出于设计简单小巧的考虑,并没有为Picorv32设计缓存。这显然增大了Picorv32在内存访问上的延时。
为了减小Picorv32在访存上的开销,计划为Picorv32处理器添加数据缓存与指令缓存。

设计一块稳定、可用的缓存会需要不少的时间进行开发,这显然会加大项目的工作量。所幸,当下存在这大量的
开源缓存可供使用。一个名为IOB的项目为本课题提供了可行的解决方案\cite{roque2021iob}。该项目提供了可行的通用户缓存
解决方案。本课题计划利用IOB项目所提供的缓存实现,对Picorv32进行修改。

在设计中,为Picorv32同时添加了指令缓存与数据缓存。同时,为了避免因数据一致性产生的问题,还应当
留出一条不经由缓存的数据旁路。设立数据旁路主要是出于处理器核与外设之间的数据一致性考虑:如果处理器核
访问外设的通过经过缓存,则处理器核访问的数据就很有可能是缓存是经过缓存的数据,而不是真实的来自外设
数据。这会导致处理器核无法及时访问到来自外设的数据。同样的道理也适用于写内存操作。

\autoref{fig:pico_no_cache}、\autoref{fig:pico_single_cache}与\autoref{fig:pico_dual_cache}分别
展示了三种不同的Picorv32处理器架构设计方案。这三种架构分别对应了不使用缓存、使用统一的一级缓存
以及使用数据缓存、指令缓存以及二级缓存的架构设计方案。

\csvgfig{pico_no_cache}{不使用缓存的Picorv32处理器架构设计}{1.0}

不使用任何缓存的设计来自于Picorv32项目本身\cite{picorv32}。这样的设计足够简单,面积与频率
都能达到较好水平。同时,因为不存在缓存,也就无需担心缓存所造成的数据一致性问题。但是缺少缓存在
性能上的影响也是非常明显的。后续的章节将会说明这一点。在本课题中,使用本设计作为对照组,与添加
缓存的设计进行性能比较。

\csvgfig{pico_single_cache}{使用单块缓存的Picorv32处理器架构设计}{1.0}

使用统一的一级缓存的好处在于结构设计上较为简单,数据通路较为简单。Picorv32在设计上指令访存与
数据访存通过统一的接口访问存储器。因此这样的设计也比较契合Picorv32本身的接口设计。但统一缓存的
缺陷也同样明显。在这样的设计下,指令与数据共用统一块缓存,互相之间发生淘汰的可能性较大,因此可能
会导致缓存出现颠簸,进而造成性能上的损失。

\csvgfig{pico_dual_cache}{使用数据缓存与指令缓存的Picorv32处理器架构设计}{1.0}

二级缓存设计是经典的通用处理器缓存架构设计方案。在指令缓存与数据缓存分离的同时,使用一块更大的
高速存储器件对数据进行更进一步的缓存处理。显然,这样的方案会带来更多的面积和功耗开销,从而加大
生产、制造成本。在一般的ASIC设计方案中通常会使用片外的高速存储器作为二级缓存,以降低芯片自身
的成本。但这样做的好处也是同样明显的。使用二级缓存可以有效地降低访问缓存时的未命中概率,指令缓存
与数据缓存分离也能一定程度上避免因为在取指操作与去取数据操作相互淘汰而造成的颠簸,避免了无谓的性能
浪费。

\subsubsection{支持访存操作的缓存队列设计}

Corundum使用的缓存队列是典型的先进先出(First In First Out,FIFO)结构,使用一组入队接口与
出队接口用于连接缓存队列前后的处理模块。这样的设计显然不能满足本课题的设计要求。为了实现本课题所
需要的功能,需要在Corundum缓冲队列的基础上增加新的端口设计。

新增加的端口主要可以分为两个部分:请求交互接口和访存接口。其中,请求交互接口连接缓存队列与调度器,
用于向调度器发送数据包处理请求,并接收来自调度器的回复信号。请求信号采用如\autoref{tbl:design_req_signal}所示
的格式。设计上使用一对握手信号来平衡缓存端口与调度器之间的数据包处理速度。此外,信号中包含了请求
所对应的数据包在缓存队列中的起始地址。使用起始地址可以唯一确定队列中的任意数据包,同时传递数据包
的起始地址也便于网络处理器后续对数据包进行寻址访存。

\begin{generaltab}{请求信号格式}{tbl:design_req_signal}
  \begin{tabular}{ccc}
    \toprule
    信号名 & 信号位宽 & 信号说明 \\
    \midrule
    req\_valid & 1 & 请求握手信号 \\
    req\_ready & 1 & 请求握手信号 \\
    req\_addr & 32 & 数据包起始地址 \\
    \bottomrule
  \end{tabular}
\end{generaltab}

新增的另一类端口是用于网络处理器访问的AXI Lite总线接口。AXI Lite端口是AXI4端口的简化版本\cite{intro_amba_axi}。使用
AXI Lite总线端口的目的在于为网络处理器提供可用的访存端口。该端口连接到XBAR互联结构,用于网络
处理器的访存操作。

\subsubsection{网络处理器调度机制设计}

\autoref{fig:design_controlpath}中给出了整个系统控制通路的设计。这其中最为关键的部件就是
负责对来自Corundum缓存队列的请求进行调度的网络处理器调度机制。该机制负责将来自缓存队列的请求
进行整理与分发,传递给其所管理的网络处理器进行处理。

调度器应用了一条缓冲队列对来自Corundum缓冲队列的请求进行缓存,以避免因为网络处理器较慢的处理
速度而导致Corundum缓冲队列停滞。请求缓冲队列在设计上与普通的缓冲队列基本上是一致的,但本课题为
其加入了一些旨在提升性能的改进。本课题在传统缓冲队列的基础上,加入了Fall-Through功能。这一功能
允许当队列为空,且入队端口与出队端口同时有效时,入队数据可以通过旁路直接从出队端口流出,而不需要
通过缓存队列内的存储单元。这样的设计可以在时序上取得更好的性能表现。

进入了调度器缓冲队列的数据请求将经由分发单元按序分发到各个网络处理器的控制单元。分发单元(Dispatcher)
是调度器的核心单元。该单元将动态地监测各个网络处理器的状态,并根据网络处理器的当前状态考虑为其
分配待处理的数据包请求。分发单元采用经典的Round Robin算法来实现网络处理器之间的负载均衡,以
确保每个网络处理器的性能都能够得到充分的利用。

控制单元是调度器中负责与网络处理器进行直接交互的单元,也就直接负责对网络处理器进行控制。在调度器
中,每一个网络处理器都对应了一个控制单元。控制单元上包含了一系列的控制寄存器,并通过这些寄存器实现
了对于网络处理器的控制。在设计上,网络处理器可以通过访存操作直接访问控制单元上的控制寄存器,从而
实现接收、应答数据包请求的功能。

\begin{generaltab}{控制单元寄存器内存布局}{tbl:design_scheduler_ctrl}
  \begin{tabular}{cccc}
    \toprule
    寄存器名 & 寄存器起始地址 & 寄存器位宽 & 寄存器说明 \\
    \midrule
    DOORBELL & 0x0000 & 32 & 门铃寄存器 \\
    PKT\_ADDR & 0x0004 & 32 & 待处理数据包的起始地址 \\
    \midrule
    NP\_BUSY & 不可访问 & 1 & 网络处理器当前状态 \\
    \bottomrule
  \end{tabular}
\end{generaltab}

当网络处理器当前处于空闲状态时,NP\_BUSY寄存器变为逻辑零,以告知分发单元该网络处理器正处于空闲
状态,等待新的数据包处理请求带来。而当数据包处理请求经由分发单元被分配到网络处理器后,数据包处理
请求中的req\_addr一项就会被写入到PKT\_ADDR寄存器中,同时,NP\_BUSY寄存器被置1,网络处理器被唤醒,
从而开始工作。通过运行时软件,网络处理器将会从PKT\_ADDR寄存器上读取到对应的数据包的起始地址,从而
开始访问Corundum缓存队列,读取数据包数据并开始进行处理。当处理完成后,运行时控制网络处理器访问
作为门铃寄存器的DOORBELL寄存器,告知控制单元当前数据包已处理完成。控制单元接收到对门铃寄存器的
访存请求后,重新将NP\_BUSY寄存器置0,使网络处理器进入休眠,从而完成了一次网络数据包处理。

\subsection{本章小结}

本章主要讨论了本课题关于松散RISC-V多核片外访存系统的整体设计。首先,本章重新梳理了本课题所面对
的项目需求,并结合项目需求,对本课题的研究内容及其可行性进行了阐述。然后,本章给出了项目的整体
架构设计,并对设计思路进行了描述。进一步的,本章对如缓冲队列、请求调度、访存架构设计等关键设计要点
作出了详细的描述。通过上述描述,本章给出了本课题设计的数据通路与控制通路设计。在下一章中,将对
这些技术要点的具体实现作出更为细节的描述,并说明在实现过程中所需要着重考虑的技术要点与技术难点。
